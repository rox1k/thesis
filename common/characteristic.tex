
{\actuality} Открытые в  2007 году керровские оптические частотные гребенки в оптических микрорезонаторах \cite{DelHaye2007,Kippenberg2011} вызвали новую революцию в области метрологии. Керровские гребенки позволяют достичь уровня миниатюризации и энергоэффективности, труднодостижимого для гребенок, полученных с помощью фемтосекундных лазеров в режиме синхронизации мод, что в свою очередь позволяет существенно уменьшить размеры генераторов гребенок и создавать их на одном чипе, что в настоящее время исследуется во множестве лабораторий.

Основными преимуществами частотных гребенок из микрорезонаторов являются их компактность, высокая мощность, приходящаяся на каждую компоненту гребенки, и возможность получения частот повторения в диапазоне 10-1000 ГГц, важном для многих приложений, включая телекоммуникации высокой пропускной способности \cite{Pfeifle2014}, астрофизику \cite{Glenday2015}, синтез частот \cite{Ferdous2011}, радиофотонику и генерацию микроволн \cite{Xue2016,Savchenkov2008}. Оптическая частотная гребёнка может быть использована как высокостабильный калибровочный репер для прецизионной спектроскопии, как источник спектрально чистого СВЧ сигнала, а также для генерации фемтосекундных оптических импульсов – генераторы гребенки могут обеспечивать короткие периодические оптические импульсы с очень малым временным джиттером. Параметрический характер усиления обеспечивает широкополосность процесса в полосе, достигающей октавы. Подобный спектр гребенки может связывать оптический диапазон с микроволновым и наоборот. Кроме того, ширина полосы усиления ограничена только окном прозрачности материала микрорезонатора, что позволяет генерировать гребенки как в УФ, так и среднем ИК диапазонах.

В последние годы наблюдается быстрый и существенный прогресс в области керровских частотных гребенок. Гребенки были продемонстрированы в различных структурах, в том числе в кристаллических резонаторах из различных фторидов \cite{Savchenkov2008,Grudinin2012,Jost2015,Liang2011,DelHaye2011,Chembo2010,Grudinin2009}, в интегральных микрорезонаторах из нитрида кремния (SiN) \cite{Levy2010,Okawachi2011,Johnson2012,Huang2015}.

% {\progress}
% Этот раздел должен быть отдельным структурным элементом по
% ГОСТ, но он, как правило, включается в описание актуальности
% темы. Нужен он отдельным структурынм элемементом или нет ---
% смотрите другие диссертации вашего совета, скорее всего не нужен.

{\aim} данной работы является поиск методов генерации оптических частотных гребенок в микрорезонаторах при различной дисперсии групповой скорости на различных длинах волн накачки, экспериментальное получение солитонов в кристаллических микрорезонаторов и изучение их возможных приложений.

Для~достижения поставленной цели необходимо было решить следующие {\tasks}:
\begin{enumerate}
  \item Разработать численный метод моделирования динамики оптических частотных гребенок в микрорезонаторах.
  \item Разработать методику изготовления кристаллических микрорезонаторов с высокой добротностью.
  \item Создать экспериментальную установку для изучения кристаллических микрорезонаторов.
  \item Экспериментально получить солитонный режим генерации керровской гребенки, изучить его свойства и продемонстрировать практическое применение.
\end{enumerate}

{\novelty}
\begin{enumerate}
  \item На основе численного моделирования был предложен оригинальный метод генерации оптических частотных гребенок при нормальной дисперсии групповой скорости в резонаторе при отклонении дисперсионного закона.
  \item На основе численного моделирования был предложен оригинальный метод генерации оптических частотных гребенок при нормальной дисперсии групповой скорости в резонаторе при использовании двухчастотной или амлитудно-модулированной накачке.
  \item Впервые была продемонстрирована методика изготовления идентичных по форме кристаллических микрорезонаторов высокой добротности с различием по диаметру в 1 мкм.
  \item Впервые продемонстрирована возможность одновременнной генерации солитонов в идентичных микрорезонаторах, расположенных на одном кристаллическом цилиндре.
  \item Впервые продемонстрирована возможность одновременнной генерации солитонов в одном резонаторе на разных семействах пространственных мод распространяющихся как в одном, так и в противоположных направлениях. 
\end{enumerate}

{\influence} разработанной методики изготовления кристаллических микрорезонаторов заключается в возможности повторяемо изготавливать высокдобротные микрорезонаторы с заданными характеристиками. Экспериментальная демонстрация генерации двойных оптических в одном резонаторе на разных семействах мод имеет прямое приложение в устройствах спектроскопии поглощения и быстрого измерения расстояний. Экспериментальная демонстрация солитонного режима имеет прямое практическое применение как источника высокостабильного СВЧ сигнала на частоте повторения солитонов.

{\methods} В работе использовались как общенаучные методы: анализ, наблюдение, сравнение, эксперимент, так и специальные методы численного компьютерного моделирования.

{\defpositions}
\begin{enumerate}
  \item Первое положение
  \item Второе положение
  \item Третье положение
  \item Четвертое положение
\end{enumerate}

{\reliability} Достоверность полученных результатов определяется адекватностью использованных физических моделей и математических методов, выбранных для решения поставленных задач, корректностью использованных приближений, а также
соответствием результатов теоретических и численных расчетов и экспериментальных данных, и не вызывает сомнений. Результаты находятся в соответствии с результатами, полученными другими авторами.

Численная модель основана на решении системы нелинейных дифференциальных уравнений, полученных из уравнений Максвелла. Использовались широко известные численные методы решения ОДУ. В основе экспериментальных исследований лежали классические методы оптики и методики измерений физических величин.

{\probation}
Основные результаты работы докладывались~на:
\begin{enumerate}
  \item Microresonator Frequency Combs and Applications, Ascona, 2014
  \item CLEO/QELS, San Jose, 2014
  \item PQE-2015, Snowbird UT, 2015
  \item CLEO, San Jose, 2015
  \item XV Всероссийская школа-семинар "Физика и применение микроволн" имени профессора А.П. Сухорукова ( "Волны-2015", Красновидово, 2015
  \item Third International Conference on Quantum Technologies (ICQT 2015), Москва, 2015
  \item Photonics West, Сан-Франциско, 2016
  \item EMN Optical Communications Meeting, Дубай, 2016
  \item SPIE Photonics West, Сан-Франциско, 2016
  \item Progress In Electromagnetics Research Symposium (PIERS 2017 in St Petersburg), Санкт-Петербург, 2017
  \item XVI Всероссийская школа-семинар "Физика и применение микроволн" имени профессора А.П. Сухорукова ( "Волны-2017"), Красновидово, 2017
  \item CLEO Europe & EQEC 2017
\end{enumerate}


{\contribution} Все результаты, вошедшие в диссертационную работу, получены либо лично
автором, либо совместно с соавторами работ, опубликованных по теме диссертации.

%\publications\ Основные результаты по теме диссертации изложены в ХХ печатных изданиях~\cite{Sokolov,Gaidaenko,Lermontov,Management},
%Х из которых изданы в журналах, рекомендованных ВАК~\cite{Sokolov,Gaidaenko},
%ХХ --- в тезисах докладов~\cite{Lermontov,Management}.

\ifnumequal{\value{bibliosel}}{0}{% Встроенная реализация с загрузкой файла через движок bibtex8
    \publications\ Основные результаты по теме диссертации изложены в 16 печатных изданиях,
    7 из которых изданы в журналах, рекомендованных ВАК,
    9 "--- в тезисах докладов.%
}{% Реализация пакетом biblatex через движок biber
%Сделана отдельная секция, чтобы не отображались в списке цитированных материалов
    \begin{refsection}[vak,papers,conf]% Подсчет и нумерация авторских работ. Засчитываются только те, которые были прописаны внутри \nocite{}.
        %Чтобы сменить порядок разделов в сгрупированном списке литературы необходимо перетасовать следующие три строчки, а также команды в разделе \newcommand*{\insertbiblioauthorgrouped} в файле biblio/biblatex.tex
        \printbibliography[heading=countauthorvak, env=countauthorvak, keyword=biblioauthorvak, section=1]%
        \printbibliography[heading=countauthorconf, env=countauthorconf, keyword=biblioauthorconf, section=1]%
        \printbibliography[heading=countauthornotvak, env=countauthornotvak, keyword=biblioauthornotvak, section=1]%
        \printbibliography[heading=countauthor, env=countauthor, keyword=biblioauthor, section=1]%
        \nocite{%Порядок перечисления в этом блоке определяет порядок вывода в списке публикаций автора
                vakbib1,vakbib2,%
                confbib1,confbib2,%
                bib1,bib2,%
        }%
        \publications\ Основные результаты по теме диссертации изложены в~\arabic{citeauthor}~печатных изданиях,
        \arabic{citeauthorvak} из которых изданы в журналах, рекомендованных ВАК,
        \arabic{
        authorconf} "--- в~тезисах докладов.
    \end{refsection}
    \begin{refsection}[vak,papers,conf]%Блок, позволяющий отобрать из всех работ автора наиболее значимые, и только их вывести в автореферате, но считать в блоке выше общее число работ
        \printbibliography[heading=countauthorvak, env=countauthorvak, keyword=biblioauthorvak, section=2]%
        \printbibliography[heading=countauthornotvak, env=countauthornotvak, keyword=biblioauthornotvak, section=2]%
        \printbibliography[heading=countauthorconf, env=countauthorconf, keyword=biblioauthorconf, section=2]%
        \printbibliography[heading=countauthor, env=countauthor, keyword=biblioauthor, section=2]%
        \nocite{vakbib2}%vak
        \nocite{bib1}%notvak
        \nocite{confbib1}%conf
    \end{refsection}
}
%При использовании пакета \verb!biblatex! для автоматического подсчёта
%количества публикаций автора по теме диссертации, необходимо
%их~здесь перечислить с использованием команды \verb!\nocite!.
