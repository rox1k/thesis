\chapter{Длинное название главы, в которой мы смотрим на~примеры того, как будут верстаться изображения и~списки} \label{chapt2}

\section{Одиночное изображение} \label{sect2_1}

\begin{figure}[ht]
  \centering
  \includegraphics [scale=0.27] {latex}
  \caption{TeX.}
  \label{img:latex}
\end{figure}

%\newpage
%============================================================================================================================
\section{Длинное название параграфа, в котором мы узнаём как сделать две картинки с~общим номером и названием} \label{sect2_2}

А это две картинки под общим номером и названием:
\begin{figure}[ht]
  \begin{minipage}[ht]{0.49\linewidth}\centering
    \includegraphics[width=0.5\linewidth]{knuth1} \\ а)
  \end{minipage}
  \hfill
  \begin{minipage}[ht]{0.49\linewidth}\centering
    \includegraphics[width=0.5\linewidth]{knuth2} \\ б)
  \end{minipage}
  \caption{Очень длинная подпись к изображению, на котором представлены две фотографии Дональда Кнута}
  \label{img:knuth}
\end{figure}

Те~же~две картинки под~общим номером и~названием, но с автоматизированной нумерацией подрисунков:
\begin{figure}[ht]
    {\centering
        \hfill
        \subbottom[List-of-Figures entry][Первый подрисунок\label{img:knuth_2_1}]{%
            \includegraphics[width=0.25\linewidth]{knuth1}}
        \hfill
        \subbottom[\label{img:knuth_2_2}]{%
            \includegraphics[width=0.25\linewidth]{knuth2}}
        \hfill
        \subbottom[Третий подрисунок]{%
            \includegraphics[width=0.3\linewidth]{example-image-c}}
        \hfill
    }
    \legend{Подрисуночный текст, описывающий обозначения, например. Согласно
    ГОСТ 2.105, пункт 4.3.1, располагается перед наименованием рисунка.}
    \caption[Этот текст попадает в названия рисунков в списке рисунков]{Очень
    длинная подпись к второму изображению, на котором представлены две
    фотографии Дональда Кнута}
    \label{img:knuth_2}
\end{figure}

На рисунке~\ref{img:knuth_2_1} показан Дональд Кнут без головного убора. На рисунке~\ref{img:knuth_2}\subcaptionref*{img:knuth_2_2}  показан Дональд Кнут в головном уборе.


\section{Пример вёрстки списков} \label{sect2_3}

\noindent Вложенные списки:
\begin{itemize}
  \item Имеется маркированный список.
  \begin{enumerate}
    \item В нём лежит нумерованный список,
    \item в котором
    \begin{itemize}
      \item лежит ещё один маркированный список.
    \end{itemize}
  \end{enumerate}
\end{itemize}


\subsection{Пробелы}

В~русском наборе принято:
\begin{itemize}
    \item единицы измерения, знак процента отделять пробелами от~числа: 10~кВт, 15~\% (согласно ГОСТ 8.417, раздел 8);
    \item $\tg 20^\circ$, но: 20~${}^\circ$C (согласно ГОСТ 8.417, раздел 8);
    \item знак номера, параграфа отделять от~числа: №~5, \S~8;
    \item стандартные сокращения: т.\:е., и~т.\:д., и~т.\:п.;
    \item неразрывные пробелы в~предложениях.
\end{itemize}

\subsection{Математические знаки и символы}


\subsection{Кавычки}
В английском языке приняты одинарные и двойные кавычки в~виде ‘...’ и~“...”. В России приняты французские («...») и~немецкие („...“) кавычки (они называются «ёлочки» и~«лапки», соответственно). <<Лапки>> обычно используются внутри ,,ёлочек``, например, <<... наш гордый ,,Варяг``...>>.

Вместо лигатур или команд с~активным символом "\ можно использовать команды \verb|\glqq| и \verb|\grqq| для набора немецких кавычек и команды \verb|\flqq| и~\verb|\frqq| для набора французских кавычек. Они определены в пакете \verb|babel|.

\subsection{Тире}
%  babel+pdflatex по умолчанию, в polyglossia надо включать опцией (и перекомпилировать с удалением временных файлов)
Команда \verb|"---| используется для печати тире в тексте. Оно несколько короче английского длинного тире. Кроме того, команда задаёт небольшую жёсткую отбивку от слова, стоящего перед тире. При этом, само тире не отрывается от~слова. После тире следует такая же отбивка от текста, как и перед тире. При наборе текста между словом и командой, за которым она следует, должен стоять пробел.

В составных словах, таких, как <<Закон Менделеева"--~Клапейрона>>, для печати тире надо использовать команду \verb|"--~|. Она ставит более короткое, по~сравнению с~английским, тире и позволяет делать переносы во втором слове. При~наборе текста команда \verb|"--~| не отделяется пробелом от слова, за которым она следует (\verb|Менделеева"--~|). Следующее за командой слово может быть  отделено от~неё пробелом или перенесено на другую строку.

Если прямая речь начинается с~абзаца, то перед началом её печатается тире командой
\verb|"--*|. Она печатает русское тире и жёсткую отбивку нужной величины перед текстом.
