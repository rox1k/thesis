\chapter{Обзор литературы} \label{chapt1}

\section{Современное состояние исследований оптических частотных гребенок в микрорезонаторах} \label{sect1_1}

Развитие фемтосекундных оптических частотных гребенок, отмеченных Нобелевской премией в 2005 г. [1], оказало огромное влияние на науку и технологии с момента их первоначального открытия в 2000 году. Помимо хорошо известных приложений в прецизионной частотной метрологии и для создания атомных часов [2-5], возможность эффективно управлять амплитудой и фазой точно определенных спектральных компонент открыло совершенно новые подходы для генерации аттосекундных импульсов [6,7] спектроскопии [8-11], обработки оптических сигналов и радиофотоники [12,13] и для многих других важных практических приложений [14-16]. Открытые в лаборатории профессора Киппенберга в 2007 году керровские частотные гребенки в оптических микрорезонаторах из плавленого кварца (рис. 1, 2) [17-18] вызвали вторую революцию в области метрологии. Керровские гребенки позволяют достичь уровня миниатюризации и энергоэффективности, труднодостижимого для гребенок, полученных с помощью фемтосекундных лазеров в режиме синхронизации мод, что в свою очередь позволяет существенно уменьшить размеры генераторов гребенок и создавать их на одном кристалле (рис. 3), что в настоящее время исследуется во множестве лабораторий, в частности, в США. Число статей, публикуемых в год по этой тематике, достигает сейчас почти 100 и продолжает расти с каждым годом (рис. 4).
Возможность синтезировать частотные гребенки непосредственно на чипе открывает путь к полностью интегрированным генераторам оптических гребенок, которые могут существенно упростить процессы измерения и синтеза частоты, и, тем самым, выйти на коммерческий уровень. Основными преимуществами микрорезонаторных частотных гребенок являются их компактность, высокая мощность, приходящаяся на каждую компоненту гребенки, и возможность получения частот повторения в диапазоне 10-100 ГГц, важном для многих приложений, включая телекоммуникации высокой пропускной способности [19], астрофизику [20], синтез частот [21], радиофотонику и генерацию микроволн [22-23]. Оптическая частотная гребёнка может быть использована как высокостабильный стандарт частоты для прецизионной спектроскопии, как источник спектрально чистого СВЧ излучения, а также для генерации фемтосекундных оптических импульсов – генераторы гребенки могут обеспечивать короткие периодические оптические импульсы (солитоны длительностью ~200 фс в кристаллических и ~20 фс в интегральных) с очень малым временным джиттером. Параметрический характер усиления обеспечивает широкополосность процесса в полосе, достигающей октавы. Подобный спектр гребенки может связывать оптический диапазон с микроволновым и наоборот. Кроме того, ширина полосы усиления ограничена только окном прозрачности, что позволяет генерировать гребенки как в УФ, так и среднем ИК диапазонах.
В последние годы наблюдается быстрый и существенный прогресс в области микрорезонаторных частотных гребенок. Гребенки были продемонстрированы в различных структурах (рис. 2), в том числе в кристаллических резонаторах [23-29], в интегральных микрорезонаторах из нитрида кремния (SiN) [30-33], в планарных системах, изготовленных из стекла Hydex [34,35], а также во многих других диэлектрических материалах [36,37]. Такие гребенки были использованы для обоснования различных концептуальных приложений, в том числе, демонстрации оптических атомных часов, синтеза оптических сигналов, демонстрации подсчета периодов световой волны, а также когерентной связи с терабитной скоростью. Недавние эксперименты швейцарской группы продемонстрировали передачу данных на расстояние 300 км с терабитной скоростью при квадратурной амплитудной модуляции [19].
Кроме того, был проведен тщательный анализ и достигнуто понимание богатой нелинейной динамики процесса формирования гребенок; объяснены механизмы появления шумов в керровских гребенках [38], а также были определены состояния, отвечающие низким уровням шума. Следует отметить, что в отличие от частотных гребенок, получаемых при синхронизации мод лазера, произвольные микрорезонаторные частотные гребенки не представляют собой сверхкороткие импульсы во временном представлении из-за произвольных фазовых соотношений между отдельными спектральными линиями, появляющимися в процессе формирования гребенки. Решить эту проблему и получить малошумные, когерентные частотные гребенки с гладким спектральным профилем можно путем генерации диссипативных керровских солитонов. Существенный вклад в этой области были сделаны совместно группами Киппенберга и Городецкого. В частности, они предложили новый метод генерации микрорезонаторных оптических гребенок на основе временных солитонов в оптических микрорезонаторах (рис. 5) [39]. Предложенный метод основан на плавном изменении частоты накачки, приводящем к формированию временных солитонов. Эти результаты были подкреплены заявкой на международный патент [40]. Этот результат стал большим прорывом в этой области, поскольку позволил генерировать когерентные и широкополосные оптические гребенки, которые могут быть достигнуты детерминировано и могут быть смоделированы с помощью уравнения Луджиато-Лефевра. Также эта работа показала возможность генерации фемтосекундных когерентных периодических оптических импульсов с малым джиттером. Недавно, с помощью этого метода солитоны были получены в кварцевых микрорезонаторах, изготавливаемых методом фотолитографии [41]. Позже были описаны механизмы, препятствующие образованию солитонов в микрорезонаторах, и методы их преодоления [42]. В недавней совместной работе EPFL/МЦКТ-МГУ показали формирование и распространение солитона и генерацию гребенки в 2/3 октавы в микрорезонаторе, интегрированном на чип [43].  В целом, в последнее время появилось множество работ, посвященных генерации и свойствам солитонов в микрорезонаторах или, что тоже самое, но в частотном представлении, солитонным (когерентным) керровским частотным гребенкам [44-54].
В то же время в последние несколько лет было предложено несколько перспективных методов, позволяющих обойтись без значительной перестройки частоты накачки: один из них базируется на использовании фазомодулированной накачки [55,56], а второй – на тепловых эффектах [57,58], в том числе и с использованием дополнительного нагревательного элемента [58]. При нагреве микрорезонатора происходит сдвиг резонансных частот относительно частоты накачки, что в какой-то мере имитирует ранее описанный процесс перестройки частоты. Также было показана возможность получения односолитонного режима, важного для практических применений. Это осуществлялось либо контролируемым нагревом и охлаждением нагревательного элемента, либо обратной перестройкой частоты [59]. Также недавно был продемонстрирован метод генерации солитонов с помощью PIN-структур, позволяющих контролировать время жизни свободных носителей заряда [60].
Одной из приоритетных задач в области микрорезонаторных частотных гребенок является расширение спектрального покрытия существующих керровских гребенок. Эта задача, несомненно, представляет широкий интерес (рис. 6) и, например, является приоритетным направлением в грантовой программе SCOUT (Spectral Combs from UV to THz – Спектральные гребенки от ультрафиолета до терагерцового диапазона), запущенной недавно в США агентством DARPA [61] с большим финансированием.
Перспективным направлением исследований является генерация керровских частотных в среднем ИК диапазоне. Этот диапазон интересен тем, что в нем находятся колебательные уровни многих молекул, из-за чего его называют диапазоном “молекулярной дактилоскопии” [9]. К таким молекулам относятся углеводороды, вещества, важные для экологии, токсичные химикаты. В настоящее время средний ИК изучается в основном с помощью инфракрасных-Фурье спектрометров (ИКФС) [62]. Мировой рынок ИК-Фурье-спектрометров оценивается более чем в 2 миллиарда евро. Оптические гребенки в среднем ИК могут представлять собой очень интересный вариант для молекулярной дактилоскопии, а компактный источник частотной гребенки в среднем ИК может обеспечить значительный потенциал для расширения возможностей и областей применения среднего ИК диапазона. С одной стороны, он может быть использован в качестве яркого источника для Фурье-спектрометра. Более интересной является перспектива использования таких компактных устройств для двухгребеночной спектроскопии [63-66], когда накладываются два источника гребенок в среднем ИК с различными частотами повторения, и генерируется сигнал биений на детекторе, что может быть использовано для восстановления профилей поглощения молекул в быстром, компактном устройстве без подвижных частей.
Тем не менее, технология генерации керровских гребенок в среднем ИК остается сравнительно слаборазвитой и количество публикаций, посвященных этой проблеме, сравнительно мало [67-71]. В частности, в экспериментах, проведенных в EPFL при использовании параметрического лазера непрерывного излучения в качестве накачки, проф. Киппенберг в сотрудничестве с Т. Хэншем и Н. Пике показали возможность генерации оптических гребенок в среднем ИК на длине волны 2.5 мкм [68]. Генерация в среднем ИК также была продемонстрирована в структурах из нитрида кремния [60, 69, 70]. Также генерация гребенки со спектральной шириной, превышающей половину октавы, была продемонстрирована в микрорезонаторах из CaF2 и MgF2 при накачке квантово-каскадным лазером [71]. В основном же для генерации гребенок в среднем ИК успешно применялись такие подходы, как параметрическая генерация или генерация разностной частоты [72-75], однако ширина полосы и величина мощности, приходящейся на каждую компоненту гребенки, а также степень когерентности остаются ограниченными так же, как и компактность экспериментальной установки. Поэтому создание эффективного компактного генератора когерентных керровских гребенок в среднем ИК является очень важной задачей. В качестве материала, пригодного для генерации частотных гребенок в среднем ИК, в последнее время широко обсуждается фторид бария [76-77] и другие фториды [78-79].
Еще одним чрезвычайно важным направлением исследований является возможность генерации когерентных гребенок в режиме нормальной дисперсии [80], то есть в видимом и ближнем ИК диапазоне длин волн (рис. 7). Что касается видимого диапазона [81, 82], следует отметить, что керровские частотные гребенки с большим межмодовым расстоянием могут, в частности, использоваться для астрофизических измерений [83, 84] и приложений рамановской спектральной визуализации. Также возможно применением керровских гребенок в видимом диапазоне и ближнем ИК для двухгребеночной спектроскопии [85]. Видимый диапазон привлекателен еще и тем, что в нем возможна визуализация биологических веществ из-за наличия в нем окна спектра пропускания воды, и он также характеризуется большими сечениями комбинационного рассеяния. Отметим, что в основном до настоящего времени для генерации частотных гребенок в видимом диапазоне использовалось совместное действие квадратичной и кубичной нелинейностей [81, 82].
Из-за наличия пиков поглощения в УФ многие диэлектрические материалы имеют нормальную дисперсию групповых скоростей (ДГС) в видимом диапазоне и ближнем ИК, что препятствует генерации солитонов. Экспериментально узкие керровские частотные гребенки при нормальной ДГС были получены в кристаллических микрорезонаторах [86-89] и микрокольцах на чипе [90]. Также бы теоретически и экспериментально существование темных временных солитонов [44, 91-94]. В частности, в ряде работ существование темных солитонов было связано с наличием волн переключения между двумя плосковолновыми состояниями [93-94]. Однако на сегодняшний день полное понимание динамики этого процесса, в том числе механизмов, обеспечивающих генерацию керровских частотных гребенок с малым шумом в режиме нормальной дисперсии, отсутствует.
Недавно в группе профессора Городецкого был численно показан новый класс солитонных импульсов, под названием “платиконы” ("platicons"), которые существуют в нормальном режиме ДГС (рис. 8,9) [92]. Было показано, что, меняя частоту накачки, можно управлять длительностью подобных импульсов. С точки зрения преобразования энергии накачки в энергию генерируемых импульсов генерация широких платиконов оказалась более эффективной, чем генерация обычных светлых солитонов. Также было обнаружено, что для нормальной дисперсии боковые линии гребенки, ближайшие к накачке, существенно более интенсивны, чем при аномальной дисперсии, что очень важно для создания радиофотонных генераторов СВЧ на основе гребенок. Описанная выше генерация платиконов была описана для систем с дефектом закона дисперсии, вызванным, например, взаимодействием мод различных семейств [42, 95] или эффектом затягивания частоты излучения лазера накачки в моду высокодобротного микрорезонатора. Было показано, что в системе с двумя связанными микрорезонаторами этим взаимодействием можно управлять [96]. Также, недавно было показано, что мягкое возбуждение платиконов возможно и без модификации закона дисперсии при двухчастотной накачке или при достаточной амплитудной модуляции накачки [97]. Применимость этой методика подтверждается тем, что ранее при ее помощи теоретически [98] и экспериментально [99] было продемонстрирована генерация керровских частотных гребенок в режиме аномальной дисперсии. Также недавно было продемонстрирована генерация частотной гребенки в оптоволокне с нормальной дисперсией при двухчастотной накачке [100].
Еще одним актуальным направлением исследований в области микрорезонаторных частотных гребенок является разработка методов увеличения их ширины и когерентности. Одним из исследуемых в последнее время способов является взаимодействие временных солитонов и дисперсионных волн (иногда по аналогии называемых, индуцированнымо солитонами черенковским излучением) [43, 101-107]. При накачке микрорезонатора в области аномальной дисперсии возможна генерация солитона. Если его длительность достаточно мала, то его спектр распространяется в область нормальной дисперсии и солитон может генерировать дисперсионную волну – пик излучения вблизи точки нулевой дисперсии (рис. 10). Формирование дисперсионной волны в присутствии солитона может способствовать уширению полосы генерации гребенки в область нормальной дисперсии без потери когерентности. Таким способом была экспериментально продемонстрирована генерация гребенки в 2/3 октавы в микрорезонаторе, интегрированном на чип при накачке на длине волны 1.55 мкм (рис. 11) [43].
Для генерации и реализации эффективного взаимодействия солитонов и дисперсионных волн должны выполняться определенные дисперсионные соотношения, поэтому разрабатывается множество методов получения оптимальной дисперсии (dispersion engineering). Дисперсия кристаллического резонатора может быть специально задана [28, 108-113], например, с помощью микроструктурирования его формы [112, 113]. В частности, было продемонстрировано, что эффективно изменять дисперсию можно путем прецизионного формирования микрометрических выступов (рис. 12) [112]. Также специально рассчитанная форма поперечного сечения микрорезонатора может обеспечить аномальную дисперсию в диапазоне, превышающем октаву [113]. Стоит, однако, заметить, что зачастую такие исследования не доходили до экспериментов по генерации широкополосных гребенок, а ограничивались только численным моделированием.
Большой интерес привлекает возможность использования эффектов рамановского и бриллюэновского рассеяния для генерации частотных гребенок и контроля их свойств. Например, активно исследуется влияние рамановского рассеяния на генерацию и свойства диссипативных солитонов в оптических микрорезонаторах (например, эффект сдвига спектрально максимума солитона, Raman induced soliton self-frequency shift) [114-119]. Также в нескольких недавних работах [120-121] исследовалась возможность генерации оптической частотной гребенки в полосе рамановского рассеяния в области нормальной дисперсии групповой скорости, где обычные методы, как было уже сказано, работают плохо. В работе [120] экспериментально была продемонстрирована генерация оптической частотной гребенки в кристаллическом резонаторе из MgF2 при накачке 1064 нм в присутствии интенсивного спектрального максимума, вызванного вынужденным комбинационным рассеянием (рис. 13). Это было интерпретировано авторами как совместное воздействие эффектов Рамана и Керра (Kerr-Raman interaction). Этот новаторский результат открывает новый путь к пути к генерации оптических частотных гребенок в области нормальной дисперсии групповой скорости. Однако теоретические выводы авторов основанные на численном решении уравнения Луджиато-Лефевра значительно расходятся с экспериментальными результатами и не позволяют в полной мере проанализировать процесс генерации гребенки и ее когерентные свойства.
Также актуальной задачей является генерация и изучение свойств частотных гребенок в средах, обладающих как кубической, так и квадратичной нелинейностями [122]. Такими средами являются, например, кристаллы ниобата лития и танталата лития. Также, как было показано недавно, квадратично-нелинейные эффекты могут проявляться и в микрорезонаторах, изготовленных из центросимметричных материалов (кремний, нитрид кремния) из-за механических напряжений и поверхностных эффектов [81].
Наличие квадратичной нелинейности позволяет наблюдать нелинейные эффекты как на первой, так и на второй гармониках накачки, что может привести к чрезвычайному уширению керровской частотной гребенки, генерируемой в такой среде, и переносу ее в другой диапазон. Ширина полосы такой гребенки превысит октаву, что чрезвычайно важно для прецизионного измерения частоты. Таким образом, например, были получены частотные гребенки в видимом диапазоне в микрорезонаторах из нитрида кремния [81] и нитрида алюминия [82]. Совместное действие квадратичной и кубичной нелинейностей могут также существенно изменить свойства диссипативных солитонов.
В недавних работах было показано, что возможна прямая генерация частотной гребенки в кристалле периодически поляризованного ниобата лития, помещенного во внешний резонатор [123]. Использование микрорезонаторов позволит увеличить добротность, что снизит порог по мощности накачки. Материалы с ненулевой квадратичной нелинейностью, в отличие от центрально-симметричных, в которых генерация керровских гребенок изучена намного более полно, обладают большим разнообразием эффектов, которые позволяют управлять процессом генерации гребенки и ее параметрами. Так кристаллы ниобата лития являются двулучепреломляющими, обладают электрооптическим, собственным пьезоэлектрическим, пироэлектрическим, эффектом, а также фотоупругим эффектом. Отдельной важной задачей является вопрос достижения условия фазового синхронизма при трехволновом взаимодействии в анизотропной среде. Стандартной методикой на данный момент является использование периодически поляризованных образцов (рис. 14), структура поляризации которых позволяет достичь квазисинхронизма. Вопрос об оптимальной структуре периодической поляризации, которая бы обеспечила оптимальную перекачку энергии, остается открытым. Отметим, что в средах с квадратичной нелинейностью условия фазового синхронизма играют такую же роль для генерации частотных гребенок, что и знак дисперсии в кубично-нелинейных кристаллах.
Важной задачей является разработка новых устройств для генерации микрорезонаторных гребенок на единственном чипе. Одной из конструкций может быть микрорезонатор, связанный через затухающее поле, например, с экситонами в квантовых ямах с электрической или оптической накачкой. Эта система может работать в режимах сильной связи, когда образуются поляритоны [124]. Поляритоны существуют в относительно узком спектральном диапазоне вблизи экситонного резонанса, где накачка и сильная поляритонная нелинейность включают процесс параметрического преобразования, который запускает генерацию широкой гребенки уже в фотонном (неполяритонном) режиме. Создание поляритонных состояний в диэлектрических волноводах и микрорезонаторных волноводах – это область активного исследования с многообещающими результатами, и реализация микроколец ожидается в ближайшем будущем. Идея связи микрорезонатора с усилителем недавно была проверена в эксперименте, в котором микрорезонатор был помещен в виток волоконного усилителя (EDFA) и была получена гребенка [125]. Недавняя теоретическая работа по поляритонным гребенкам рассматривала модель узкой резонансной полосы частот, где могут существовать лишь несколько спектральных линий [126].
