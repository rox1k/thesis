
{\actuality} Открытие в 2007 году оптических частотных гребенок в оптических микрорезонаторах \cite{DelHaye2007} и последующая демонстрация оптических временных солитонов \cite{Herr2014} с низким уровнем фазового шума в 2014 году вызвали новую волну интереса и исследований в области микрорезонаторов и частотной метрологии \cite{Kippenberg2011,Kippenbergeaan8083}. Оптическая частотная гребенка в микрорезонаторе представляет собой набор эквидистантных спектральных линий, получающихся каскадно при накачке микрорезонатора из нелинейного материала с помощью лазера непрерывной мощности. Керровские частотные гребенки имеют расстояния между линиями 5-1000 ГГц и позволяют достичь уровня минитюаризации и энергоэффективности устройств, труднодостижимого для гребенок, полученных с помощью фемтосекундных лазеров с синхронизации мод. В последние годы наблюдается быстрый и существенный прогресс в этой области. За прошедшее с их открытия время был проведен теоретический анализ и обширное численное моделирование богатой нелинейной динамики процесса формирования оптических гребенок. Солитонный режим оптических гребенок был продемонстрирован в кристаллических микрорезонаторах из $MgF_2$ \cite{Herr2014}, в интегральных резонаторах из $Si_3N_4$ \cite{Brasch2016}, $Si$ \cite{Yu2016}, $SiO_2$ \cite{Yi2015}, $LiNbO_3$ \cite{2018arXiv181209610H}. Ширина оптической гребенки из интегрального микрорезонатора достигла октавы \cite{Pfeiffer:17}, солитоны были продемонстрированы в ближнем ИК \cite{Karpov2018}, телекоммуникационном диапазоне \cite{Herr2014} и среднем ИК \cite{Griffith2016}. Оптические частотные гребенки в микрорезонаторах обладают богатой нелинейной динамикой и привели к наблюдению различных эффектов: генерация светлых солитонов \cite{Herr2014}, темных импульсов \cite{Xue2015}, излучение дисперсионных волн \cite{Brasch2016}, Рамановское самовоздействие и сдвиг частоты \cite{Karpov2016}, Стоксовы солитоны \cite{Yang2016stokes}, бризерные режимы \cite{Lucas2017breather}, солитонные кристаллы \cite{Cole2017crystals}. Были проведены эксперименты, демонстрирующие их разнообразные практические применения: калибровка эшелле-спектрографов для поиска экзопланет \cite{Obrzud2019}, прямая спектроскопия поглощения веществ методом двойной гребенки \cite{Suh2016}, оптические часы \cite{Papp2014}, как источник каналов в телекоммуникации с рекордной пропускной способностью \cite{MarinPalomo2017}, как компактный радиофотонный источник СВЧ сигнала с низким уровнем фазового шума \cite{Liang2015}, как источник импульсов для быстрого измерения расстояний с высокой точностью (ЛИДАР) \cite{Trocha887}, как полностью интегральный синтезатор оптических частот \cite{Spencer2018}.

Количество публикаций с 2014 по 2018 годы с ключевыми словам солитоны в оптических резонаторах составило около 600, из них несколько десятков в самых высокорейтинговых научных журналах. В случае преодоления нескольких нерешенных проблем, оптические частотные гребенки из микрорезонаторов могут стать востребованным коммерческим продуктом.

Несмотря на обширное изучение оптических частотных гребенок и солитонов в микрорезонаторах, остаются нерешенными важные задачи: генерации оптической гребенки шириной в октаву с межмодовым расстоянием <100 ГГц, доступном для детектирования современной электроникой и полная ее стабилизация; полная интеграция генератора гребенки на чип, включая лазер накачки; повышение энергетической эффективности процесса нелинейного преобразования частот, повышение мощности оптической гребенки; детерминированная генерация оптических гребенок с малыми шумами при нормальной дисперсии групповой скорости, детерминированная генерация односолитонных режимов как в одном, так одновременно и в нескольких резонаторах.


% {\progress}
% Этот раздел должен быть отдельным структурным элементом по
% ГОСТ, но он, как правило, включается в описание актуальности
% темы. Нужен он отдельным структурынм элемементом или нет ---
% смотрите другие диссертации вашего совета, скорее всего не нужен.

{\aim} данной работы является поиск методов генерации оптических частотных гребенок в микрорезонаторах при различной дисперсии групповой скорости на различных длинах волн накачки, экспериментальное получение солитонов в кристаллических микрорезонаторах и изучение их возможных приложений.

Для~достижения поставленной цели необходимо было решить следующие {\tasks}:
\begin{enumerate}
  \item Разработать численный метод моделирования динамики оптических частотных гребенок в микрорезонаторах.
  \item Разработать методику изготовления кристаллических микрорезонаторов с высокой добротностью.
  \item Создать экспериментальную установку для изучения свойств кристаллических микрорезонаторов.
  \item Экспериментально получить солитонный режим генерации керровской гребенки, изучить его свойства и продемонстрировать практические применения.
\end{enumerate}


{\novelty}
\begin{enumerate}
  \item На основе численного моделирования были указаны ограничения на ширину оптической гребенки в зависимости от дисперсии групповой скорости резонатора и на влияние эффекта нормального расщепления мод на возможность генерации солитонов.
  \item На основе численного моделирования был предложен оригинальный метод генерации оптических частотных гребенок при нормальной дисперсии групповой скорости в резонаторе при отклонении дисперсионного закона.
  \item На основе численного моделирования был предложен оригинальный метод генерации оптических частотных гребенок при нормальной дисперсии групповой скорости в резонаторе при использовании двухчастотной или амлитудно-модулированной накачки.
  \item Впервые была продемонстрирована методика изготовления идентичных по форме кристаллических микрорезонаторов высокой добротности с различием по диаметру в 1 мкм.
  \item Впервые продемонстрирована возможность одновременнной генерации солитонов в идентичных микрорезонаторах, расположенных на одном кристаллическом цилиндре.
  \item Впервые продемонстрирована возможность одновременнной генерации солитонов в одном резонаторе на разных семействах пространственных мод, распространяющихся как в одном, так и в противоположных направлениях. Продемонстрирована применимость метода для спектроскопии поглощения веществ.
  \item Впервые продемонстрирована возможность долгосрочной стабилизации частоты повторения солитона с помощью затягивания на линию амплитудной модуляции лазера
\end{enumerate}

{\influence} разработанной методики изготовления кристаллических микрорезонаторов заключается в возможности повторяемо изготавливать высокодобротные микрорезонаторы с заданными характеристиками. Экспериментальная демонстрация генерации двойных оптических гребенок в одном резонаторе на разных семействах мод имеет прямое приложение в устройствах спектроскопии поглощения и быстрого измерения расстояний. Экспериментальная демонстрация стабилизированного солитонного режима имеет прямое практическое применение как источника высокостабильного СВЧ сигнала на частоте повторения солитона.

{\methods} В работе использовались как общенаучные методы: анализ, наблюдение, сравнение, эксперимент, так и специальные методы численного компьютерного моделирования.

{\defpositions}
\begin{enumerate}
  \item Численное моделирование, основанное на решении системы уравнений для нелинейно связанных мод, дает точную картину процесса генерации оптических частотоных гребенок и солитонов в резонаторах из материалов с кубичной нелинейностью. Результаты численного моделирования очень хорошо совпадают с экспериментальными данными по режимам генерации оптических частотных гребенок и их свойствам.
  \item Использование двухчастотной или амплитудно-модулированной накачки с разницей частот, кратной области свободной дисперсии резонатора, приводит к возможности генерации темных солитоноподобных структур в микрорезонаторах с нормальной дисперсией групповой скорости.
  \item Разработанный метод изготовления кристаллических микрорезонаторов с помощью алмазного точения единственной точкой и полировки алмазными суспензиями позволяет воспроизводимо изготавливать идентичные по форме микрорезонаторы сверхвысокой добротности с различием по диаметру не более 1 мкм (разницей ОСД до 2 МГц).
  \item В двух микрорезонаторах из $MgF_2$ оптимизированной формы, изготовленных на одном кристаллическом цилиндре, возможна воспроизводимая одновременная генерация оптических временных солитонов, обладающих узкими СВЧ сигналами биений так, что разность частот повторений солитонов в этих двух микрорезонаторах не превышает 2 МГц.
  %\item В высокодобротных микрорезонаторах из $MgF_2$ возможно воспроизводимо генерировать оптические временные солитоны при аномальной дисперсии групповой скорости, обладающие узким СВЧ сигналом на частоте повторения.
  \item Высокодобротные кристаллические микрорезонаторы c МШГ из $MgF_2$ поддерживают одновременную генерацию оптических временных солитонов на разных пространственных семействах мод, распространяющихся как в одном, так и в противоположных направлениях. Результирующий СВЧ сигнал мультигетеродинирования двух оптических солитонов может использоваться для проведения спектроскопии поглощения веществ для оптических линий, находящихся в области спектрального покрытия солитонов.
  \item Высокодобротные кристаллические микрорезонаторы c МШГ из $MgF_2$ поддерживают одновременную генерацию оптических временных солитонов на разных пространственных семействах мод, распространяющихся как в одном, так и в противоположных направлениях. Результирующий СВЧ сигнал мультигетеродинирования двух оптических солитонов может использоваться для проведения спектроскопии поглощения веществ для оптических линий, находящихся в области спектрального покрытия солитонов.
\end{enumerate}

{\reliability} полученных результатов определяется адекватностью использованных физических моделей и математических методов, выбранных для решения поставленных задач, корректностью использованных приближений, а также
соответствием результатов теоретических и численных расчетов и экспериментальных данных, и не вызывает сомнений. Результаты находятся в соответствии с результатами, полученными другими авторами.

Численная модель основана на решении системы нелинейных дифференциальных уравнений, полученных из уравнений Максвелла. Использовались широко известные численные методы решения ОДУ. В основе экспериментальных исследований лежали классические методы оптики и методики измерений физических величин.

{\probation}
Основные результаты работы докладывались~на:
\begin{enumerate}
  \item Microresonator Frequency Combs and Applications, Ascona, 2014
  \item CLEO/QELS, San Jose, 2014
  \item PQE-2015, Snowbird UT, 2015
  \item CLEO, San Jose, 2015
  \item XV Всероссийская школа-семинар "Физика и применение микроволн" имени профессора А.П. Сухорукова ( "Волны-2015"), Красновидово, 2015
  \item Third International Conference on Quantum Technologies (ICQT 2015), Москва, 2015
  \item Photonics West, Сан-Франциско, 2016
  \item EMN Optical Communications Meeting, Дубай, 2016
  \item SPIE Photonics West, Сан-Франциско, 2016
  \item Progress In Electromagnetics Research Symposium (PIERS 2017 in St Petersburg), Санкт-Петербург, 2017
  \item XVI Всероссийская школа-семинар "Физика и применение микроволн" имени профессора А.П. Сухорукова ( "Волны-2017"), Красновидово, 2017
  \item CLEO Europe and EQEC, Мюнхен, 2017
\end{enumerate}


{\contribution} Все результаты, вошедшие в диссертацию, получены либо лично
автором, либо совместно с соавторами работ, опубликованных по теме диссертации.

%\publications\ Основные результаты по теме диссертации изложены в ХХ печатных изданиях~\cite{Sokolov,Gaidaenko,Lermontov,Management},
%Х из которых изданы в журналах, рекомендованных ВАК~\cite{Sokolov,Gaidaenko},
%ХХ --- в тезисах докладов~\cite{Lermontov,Management}.

\ifnumequal{\value{bibliosel}}{0}{% Встроенная реализация с загрузкой файла через движок bibtex8
    \publications\ Основные результаты по теме диссертации изложены в 20 печатных изданиях,
    11 из которых изданы в журналах, рекомендованных ВАК,
    9 "--- в тезисах докладов.%
}{% Реализация пакетом biblatex через движок biber
%Сделана отдельная секция, чтобы не отображались в списке цитированных материалов
    \begin{refsection}[vak,papers,conf]% Подсчет и нумерация авторских работ. Засчитываются только те, которые были прописаны внутри \nocite{}.
        %Чтобы сменить порядок разделов в сгрупированном списке литературы необходимо перетасовать следующие три строчки, а также команды в разделе \newcommand*{\insertbiblioauthorgrouped} в файле biblio/biblatex.tex
        \printbibliography[heading=countauthorvak, env=countauthorvak, keyword=biblioauthorvak, section=1]%
        \printbibliography[heading=countauthorconf, env=countauthorconf, keyword=biblioauthorconf, section=1]%
        \printbibliography[heading=countauthornotvak, env=countauthornotvak, keyword=biblioauthornotvak, section=1]%
        \printbibliography[heading=countauthor, env=countauthor, keyword=biblioauthor, section=1]%
        \nocite{%Порядок перечисления в этом блоке определяет порядок вывода в списке публикаций автора
                vakbib1,vakbib2,%
                confbib1,confbib2,%
                bib1,bib2,%
        }%
        \publications\ Основные результаты по теме диссертации изложены в~\arabic{citeauthor}~печатных изданиях,
        \arabic{citeauthorvak} из которых изданы в журналах, рекомендованных ВАК,
        \arabic{
        authorconf} "--- в~тезисах докладов.
    \end{refsection}
    \begin{refsection}[vak,papers,conf]%Блок, позволяющий отобрать из всех работ автора наиболее значимые, и только их вывести в автореферате, но считать в блоке выше общее число работ
        \printbibliography[heading=countauthorvak, env=countauthorvak, keyword=biblioauthorvak, section=2]%
        \printbibliography[heading=countauthornotvak, env=countauthornotvak, keyword=biblioauthornotvak, section=2]%
        \printbibliography[heading=countauthorconf, env=countauthorconf, keyword=biblioauthorconf, section=2]%
        \printbibliography[heading=countauthor, env=countauthor, keyword=biblioauthor, section=2]%
        \nocite{vakbib2}%vak
        \nocite{bib1}%notvak
        \nocite{confbib1}%conf
    \end{refsection}
}
%При использовании пакета \verb!biblatex! для автоматического подсчёта
%количества публикаций автора по теме диссертации, необходимо
%их~здесь перечислить с использованием команды \verb!\nocite!.
