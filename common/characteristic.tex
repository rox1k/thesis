{\actuality} Открытие в 2007 году оптических частотных гребенок в оптических микрорезонаторах \cite{DelHaye2007} и последующая демонстрация в 2014 году оптических временных солитонов \cite{Herr2014} с низким уровнем фазового шума вызвали волну интереса и исследований в области микрорезонаторов и частотной метрологии \cite{Kippenberg2011,Kippenbergeaan8083}. Оптическая частотная гребенка в микрорезонаторе (далее также называемая керровской частотной гребенкой или оптической гребенкой) представляет собой набор эквидистантных спектральных линий, получающихся каскадно при накачке микрорезонатора из нелинейного материала с помощью лазера непрерывного излучения. Керровские частотные гребенки имеют расстояния между линиями 5-1000 ГГц и позволяют достичь минимальных габаритов устройств, недостижимых для оптических гребенок, полученных с помощью фемтосекундных лазеров с синхронизацией мод. В последние годы наблюдается быстрый и значительный прогресс в этой области. За прошедшее с открытия время был проведен теоретический анализ и обширное численное моделирование богатой нелинейной динамики процесса формирования оптических гребенок. Солитонный режим оптических гребенок был продемонстрирован в кристаллических резонаторах из MgF$_2$ \cite{Herr2014}, в интегральных микрорезонаторах из Si$_3$N$_4$ \cite{Brasch2016}, Si \cite{Yu2016}, SiO$_2$ \cite{Yi2015}, LiNbO$_3$ \cite{2018arXiv181209610H}. Ширина оптической гребенки из интегрального микрорезонатора достигла октавы \cite{Pfeiffer:17}, солитоны были продемонстрированы в ближнем ИК на границе видимого \cite{Karpov2018}, телекоммуникационном диапазоне на 1.55 мкм \cite{Herr2014} и среднем ИК \cite{Griffith2016}. Оптические частотные гребенки в микрорезонаторах характеризуются сложной динамикой и привели к наблюдению различных эффектов: генерации светлых солитонов \cite{Herr2014}, темных солитоноподобных импульсов \cite{Xue2015}, излучения дисперсионных волн \cite{Brasch2016}, Рамановского самовоздействия и сдвига частоты \cite{Karpov2016}, Стоксовых солитонов \cite{Yang2016stokes}, бризерных режимов \cite{Lucas2017breather}, солитонных кристаллов \cite{Cole2017crystals}. Были проведены эксперименты, демонстрирующие их разнообразные практические применения: калибровка эшелле-спектрографов для поиска экзопланет \cite{Obrzud2019}, прямая спектроскопия поглощения веществ методом двойной гребенки \cite{Suh2016}, оптические часы \cite{Papp2014}, источник каналов в телекоммуникации с рекордной пропускной способностью \cite{MarinPalomo2017}, компактный радиофотонный источник СВЧ сигнала с низким уровнем фазового шума \cite{Liang2015}, источник импульсов для быстрого измерения расстояний с высокой точностью (ЛИДАР) \cite{Trocha887}, полностью интегральный синтезатор оптических частот \cite{Spencer2018}.

Количество публикаций с 2014 по 2018 годы с ключевыми словами "солитоны в оптических резонаторах" составило около 600, из них несколько десятков в самых высокорейтинговых научных журналах. В случае преодоления нескольких нерешенных проблем, оптические частотные гребенки из микрорезонаторов могут стать востребованным коммерческим продуктом.

Несмотря на обширное изучение оптических частотных гребенок и солитонов в микрорезонаторах, остаются нерешенными важные задачи: генерация оптической гребенки шириной в октаву с расстоянием между линиями менее 100 ГГц, доступном для детектирования современной электроникой и полная ее стабилизация; создание генератора гребенки в полностью интегральном исполнении, включая лазер накачки; повышение энергетической эффективности процесса нелинейного преобразования частот, повышение мощности оптической гребенки; детерминированная генерация оптических гребенок с малыми шумами при нормальной дисперсии групповой скорости резонатора, детерминированная генерация односолитонных режимов как в одном, так одновременно и в нескольких резонаторах.


% {\progress}
% Этот раздел должен быть отдельным структурным элементом по
% ГОСТ, но он, как правило, включается в описание актуальности
% темы. Нужен он отдельным структурынм элемементом или нет ---
% смотрите другие диссертации вашего совета, скорее всего не нужен.

{\aim} данной работы является поиск методов генерации оптических частотных гребенок в микрорезонаторах при различной дисперсии групповой скорости на различных длинах волн накачки, экспериментальное получение солитонов в кристаллических микрорезонаторах и изучение их возможных приложений.

Для~достижения поставленной цели необходимо было решить следующие {\tasks}:
\begin{enumerate}
  \item Разработать численный метод моделирования динамики оптических частотных гребенок в микрорезонаторах.
  \item Разработать методику изготовления кристаллических микрорезонаторов с высокой добротностью.
  \item Создать экспериментальную установку для изучения свойств кристаллических микрорезонаторов.
  \item Экспериментально получить солитонный режим генерации керровской гребенки, изучить его свойства и продемонстрировать практические применения.
\end{enumerate}


{\novelty}
\begin{enumerate}
  \item На основе численного моделирования были указаны ограничения на ширину оптической гребенки в зависимости от дисперсии групповой скорости резонатора и на влияние эффекта нормального расщепления мод на возможность генерации солитонов.
  \item На основе численного моделирования был предложен оригинальный метод генерации оптических частотных гребенок при нормальной дисперсии групповой скорости в резонаторе при отклонении дисперсионного закона от параболического вида.
  \item На основе численного моделирования был предложен оригинальный метод генерации оптических частотных гребенок при нормальной дисперсии групповой скорости в резонаторе при использовании двухчастотной или амлитудно-модулированной накачки.
  \item Впервые была продемонстрирована методика изготовления идентичных по форме кристаллических микрорезонаторов высокой добротности с различием по диаметру не более 1 мкм.
  \item Впервые продемонстрирована возможность одновременной генерации солитонов в идентичных микрорезонаторах, расположенных на одном кристаллическом цилиндре.
  \item Впервые продемонстрирована возможность одновременной генерации нескольких солитонов в одном резонаторе на разных семействах пространственных мод, распространяющихся как в одном, так и в противоположных направлениях. Продемонстрирована применимость данного метода для спектроскопии поглощения веществ.
  \item Впервые продемонстрирована возможность долговременной стабилизации частоты повторения солитона с помощью эффекта захватывания на боковую линию амплитудной модуляции лазера накачки.
\end{enumerate}

{\influence} разработанной методики изготовления кристаллических микрорезонаторов заключается в возможности повторяемо изготавливать высокодобротные микрорезонаторы с заданными характеристиками. Экспериментальная демонстрация генерации двойных оптических гребенок в одном резонаторе на разных семействах мод имеет прямое приложение в устройствах спектроскопии поглощения и устройствах быстрого измерения расстояний. Экспериментальная демонстрация стабилизированного солитонного режима имеет прямое практическое применение как источника высокостабильного СВЧ сигнала на частоте повторения солитона.

{\methods} В работе использовались как общенаучные методы: анализ, наблюдение, сравнение, эксперимент, так и специальные методы численного компьютерного моделирования.

{\defpositions}
\begin{enumerate}
  \item Разработанный вычислительный пакет программ, основанный на решении системы уравнений для нелинейно связанных мод, дает точную картину процесса генерации оптических частотоных гребенок и солитонов в резонаторах из материалов с кубичной нелинейностью. Результаты численного моделирования хорошо согласуются с экспериментальными данными по режимам генерации оптических частотных гребенок и их свойствам.
  \item Использование двухчастотной или амплитудно-модулированной накачки с разницей частот, кратной области свободной дисперсии резонатора, приводит к возможности генерации темных солитоноподобных структур в микрорезонаторах с нормальной дисперсией групповой скорости.
  \item Разработанный метод изготовления кристаллических микрорезонаторов с помощью алмазного точения и полировки алмазными суспензиями позволяет воспроизводимо изготавливать идентичные по форме микрорезонаторы с добротностью более $5\times10^8$ с различием по диаметру не более 1 мкм.
  \item В двух микрорезонаторах из MgF$_2$ оптимизированной формы, изготовленных на одном кристаллическом цилиндре, возможна воспроизводимая одновременная генерация оптических временных солитонов, обладающих узкими СВЧ сигналами биений так, что разность частот повторений солитонов в этих двух микрорезонаторах не превышает 2 МГц.
  %\item В высокодобротных микрорезонаторах из $MgF_2$ возможно воспроизводимо генерировать оптические временные солитоны при аномальной дисперсии групповой скорости, обладающие узким СВЧ сигналом на частоте повторения.
  \item Высокодобротные кристаллические микрорезонаторы из MgF$_2$ поддерживают одновременную генерацию оптических временных солитонов на разных пространственных семействах мод, распространяющихся как в одном, так и в противоположных направлениях. Результирующий СВЧ сигнал мультигетеродинирования двух оптических солитонов может использоваться для проведения спектроскопии поглощения веществ для оптических линий, находящихся в области спектрального покрытия солитонов.
\end{enumerate}

{\reliability} полученных результатов определяется адекватностью использованных физических моделей и математических методов, выбранных для решения поставленных задач, корректностью использованных приближений, а также
соответствием результатов теоретических и численных расчетов и экспериментальных данных. Результаты находятся в соответствии с результатами, полученными другими авторами.

Численная модель основана на решении системы нелинейных дифференциальных уравнений, полученных из уравнений Максвелла. Использовались широко известные численные методы решения обыкновенных дифференциальных уравнений. В основе экспериментальных исследований лежали классические методы оптики и методики измерений физических величин.

{\probation}
Основные результаты работы докладывались~на:
\begin{enumerate}
  \item Microresonator Frequency Combs and Applications, Аскона, Швейцария 2014
  \item CLEO/QELS US, Сан-Хосе, США 2014
  \item PQE-2015, Сноуберд, США 2015
  \item CLEO US, Сан-Хосе, США 2015
  \item XV Всероссийская школа-семинар "Физика и применение микроволн" имени профессора А.П. Сухорукова ("Волны-2015"), Красновидово, Россия 2015
  \item Third International Conference on Quantum Technologies (ICQT 2015), Москва, Россия 2015
  \item SPIE Photonics West, Сан-Франциско, США 2016
  \item 2016 International Conference Laser Optics (ICLO), Санкт-Петербург, Россия 2016
  \item SPIE Photonics West, Сан-Франциско, США 2017
  \item Progress In Electromagnetics Research Symposium (PIERS 2017 in St Petersburg), Санкт-Петербург, Россия 2017
  \item XVI Всероссийская школа-семинар "Физика и применение микроволн" имени профессора А.П. Сухорукова ("Волны-2017"), Красновидово, Россия 2017
  \item CLEO Europe and EQEC, Мюнхен, Германия 2017
  \item XVII Всероссийская школа-семинар "Физика и применение микроволн" имени профессора А.П. Сухорукова ("Волны-2018"), Красновидово, Россия 2018
  \item Photonics West, Сан-Франциско, США 2018
  \item CLEO US, Сан-Хосе, США 2018
  \item 2018 International Conference Laser Optics (ICLO), Санкт-Петербург, Россия 2018
  \item European Frequency and Time Forum, Турин, Италия 2018
  \item CLEO Pacific Rim 2018, Гонконг 2018
  \item Frontier in Optics 2018, Вашингтон, США 2018
\end{enumerate}


{\contribution} Все результаты, вошедшие в диссертационную работу, получены либо лично автором, либо совместно с соавторами работ, опубликованных по теме диссертации, при этом вклад автора был определяющим или равным с основным соавтором.

%\publications\ Основные результаты по теме диссертации изложены в ХХ печатных изданиях~\cite{Sokolov,Gaidaenko,Lermontov,Management},
%Х из которых изданы в журналах, рекомендованных ВАК~\cite{Sokolov,Gaidaenko},
%ХХ --- в тезисах докладов~\cite{Lermontov,Management}.

\ifnumequal{\value{bibliosel}}{0}{% Встроенная реализация с загрузкой файла через движок bibtex8
    \publications\ Основные результаты по теме диссертации изложены в 11 печатных изданиях,
    11 из которых изданы в журналах, индексируемых в базах данных Scopus и Web of Science.%

%\begin{enumerate}
%  \item Herr T., Brasch V., Jost J.D., Mirgorodskiy I., Lihachev G., Gorodetsky M.L., Kippenberg T.J. Mode spectrum and temporal soliton formation in optical microresonators // Phys. Rev. Lett. — 2014. — Vol. 113. — P. 123901
%  \item Lobanov V.E., Lihachev G., Kippenberg T.J., Gorodetsky M.L. Frequency combs and platicons in optical microresonators with normal GVD // Opt. Express. — 2015. — Vol. 23. — Pp. 7713–772
%  \item V.E. Lobanov, G. Lihachev, M.L. Gorodetsky. Generation of platicons and frequency combs in optical microresonators with normal GVD by modulated pump // EPL. — 2015. — Vol. 112. — P. 54008.
%  \item Lobanov V.E., Lihachev G.V., Pavlov N.G. et al. Harmonization of chaos into a soliton in Kerr frequency combs // Optics Express. — 2016. — Vol. 24, no. 24.— Pp. 27382–27394.
%  \item Brasch V., Geiselmann M., Herr T., Lihachev G., Pfeiffer M.H.P, Gorodetsky M.L., Kippenberg T.J. Photonic chip based optical frequency comb using soliton induced Cherenkov radiation // Science. — 2016. — Vol. 351, no. 6271. — Pp. 357–360.
%  \item Guo H., Karpov M., Lucas E., Kordts A., Pfeiffer M.H.P, Brasch V., Lihachev G., Lobanov V.E., Gorodetsky M.L., Kippenberg T.J. Universal dynamics and deterministic switching of dissipative Kerr solitons in optical microresonators // Nature Phys. — 2017. — Vol. 13, no. 1. — P. 94–102
%  \item Pavlov N.G., Lihachev G., Koptyaev S., Lucas E., Karpov M., Kondratiev N.M., Bilenko I.A., Kippenberg T.J., Gorodetsky M.L. Soliton dual frequency combs in crystalline microresonators // Optics Lett. — 2017. — Vol. 42, no. 3. — Pp. 514–517.
%  \item M. Anderson, N. G. Pavlov, J. D. Jost, G. Lihachev, J. Liu, T. Morais, M. Zervas, M. L. Gorodetsky, T. J. Kippenberg. Highly efficient coupling of crystalline microresonators to integrated photonic waveguides // Opt. Lett. — 2018. — Vol. 43, no. 9. — Pp. 2106–2109
%  \item E. Lucas, G. Lihachev, R. Bouchand et al. Spatial multiplexing of soliton microcombs // Nature Photonics. — 2018. — Vol. 12, no. 11. — Pp. 699–705.
%  \item N. G. Pavlov, S. Koptyaev, G. V. Lihachev et al. Narrow-linewidth lasing and soliton Kerr microcombs with ordinary laser diodes // Nature Photonics. — 2018. — Vol. 12, no. 11. — Pp. 694–698
%  \item W. Weng, E. Lucas, G. Lihachev et al.  Spectral Purification of Microwave Signals with Disciplined Dissipative Kerr Solitons // Phys. Rev. Lett. — 2019. — Vol. 122. — P. 013902
%\end{enumerate}


}{% Реализация пакетом biblatex через движок biber
%Сделана отдельная секция, чтобы не отображались в списке цитированных материалов
    \begin{refsection}[vak,papers,conf]% Подсчет и нумерация авторских работ. Засчитываются только те, которые были прописаны внутри \nocite{}.
        %Чтобы сменить порядок разделов в сгрупированном списке литературы необходимо перетасовать следующие три строчки, а также команды в разделе \newcommand*{\insertbiblioauthorgrouped} в файле biblio/biblatex.tex
        \printbibliography[heading=countauthorvak, env=countauthorvak, keyword=biblioauthorvak, section=1]%
        \printbibliography[heading=countauthorconf, env=countauthorconf, keyword=biblioauthorconf, section=1]%
        \printbibliography[heading=countauthornotvak, env=countauthornotvak, keyword=biblioauthornotvak, section=1]%
        \printbibliography[heading=countauthor, env=countauthor, keyword=biblioauthor, section=1]%
        \nocite{%Порядок перечисления в этом блоке определяет порядок вывода в списке публикаций автора
                HerrPRL2014,
                Brasch2016,
                Lobanov2015,
                Lobanov2015epl,
                Lobanov2016,
                Pavlov2017,
                Karpov2017,
                Anderson:18,
                Lucas2018,
                Pavlov2018,
                PhysRevLett.122.013902
            }%
        \publications\ Основные результаты по теме диссертации изложены в~\arabic{citeauthor}~печатных изданиях,
        \arabic{citeauthorvak} из которых изданы в журналах, рекомендованных ВАК,
        \arabic{
        authorconf} "--- в~тезисах докладов.
    \end{refsection}
    %\begin{refsection}[vak,papers,conf]%Блок, позволяющий отобрать из всех работ автора наиболее значимые, и только их вывести в автореферате, но считать в блоке выше общее число работ
%        \printbibliography[heading=countauthorvak, env=countauthorvak, keyword=biblioauthorvak, section=2]%
%        \printbibliography[heading=countauthornotvak, env=countauthornotvak, keyword=biblioauthornotvak, section=2]%
%        \printbibliography[heading=countauthorconf, env=countauthorconf, keyword=biblioauthorconf, section=2]%
%        \printbibliography[heading=countauthor, env=countauthor, keyword=biblioauthor, section=2]%
%        \nocite{vakbib2}%vak
%        \nocite{bib1}%notvak
%        \nocite{confbib1}%conf
%    \end{refsection}
}
%При использовании пакета \verb!biblatex! для автоматического подсчёта
%количества публикаций автора по теме диссертации, необходимо
%их~здесь перечислить с использованием команды \verb!\nocite!.
