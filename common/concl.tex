%% Согласно ГОСТ Р 7.0.11-2011:
%% 5.3.3 В заключении диссертации излагают итоги выполненного исследования, рекомендации, перспективы дальнейшей разработки темы.
%% 9.2.3 В заключении автореферата диссертации излагают итоги данного исследования, рекомендации и перспективы дальнейшей разработки темы.
\begin{enumerate}
  \item Численные исследования модели уравнений связанных мод показали, что в высокодобротных микрорезонаторах возможна генерация оптических временных солитонов, были изучены диапазоны параметров и условия, влияющие на их эффективную генерацию.
  \item Математическое моделирование показало, что при нормальной ДГС резонатора возможна генерация темных солитоноподобных структур при условии наличия изменения в законе дисперсии или при использовании двухчастотной или амплитудно-модулированной накачки.
  \item Для выполнения экспериментальных исследований была разработана методика изготовления кристаллических микрорезонаторов методом алмазного точения и полировки алмазными суспензиями. Она позволила изготовить высокодобротные резонаторы из различных материалов оптимизированной геометрии для работы с различными элементами связи.
  \item Экспериментально была продемонстрирована генерация оптических солитонов в резонаторах из $MgF_2$ с ОСД от $8.5$ до $27$ ГГц. При активной стабилизации температуры и отстройки частоты лазера накачки солитон существовал длительное время, достаточное для экспериментальной демонстрации применений.
  \item Впервые продемонстрирована возможность одновременнной генерации солитонов в идентичных микрорезонаторах, расположенных на одном кристаллическом цилиндре.
  \item Впервые продемонстрирована возможность одновременнной генерации солитонов в одном резонаторе на разных семействах пространственных мод, распространяющихся как в одном, так и в противоположных направлениях. Показана применимость метода для спектроскопии поглощения веществ.
  \item Продемонстрирован метод стабилизации частоты повторения солитона с помощью затягивания на боковую линию амплитудной модуляции лазера. 
  \item Для дальнейших исследований в данной области важной задачей является демонстрация октавных гребенок в солитонном режиме с межмодовым расстоянием, доступном для детектирования современной электроникой, а также оптических частотных гребенок в резонаторах с нормальной дисперсией. Из новых практических применений возможна демонстрация фотонного АЦП с использованием двойных оптических гребенок.
\end{enumerate}
