%%% Основные сведения %%%
\newcommand{\thesisAuthorLastName}{\todo{Лихачёв}}
\newcommand{\thesisAuthorOtherNames}{\todo{Григорий Васильевич}}
\newcommand{\thesisAuthorInitials}{\todo{Г.\,В.}}
\newcommand{\thesisAuthor}             % Диссертация, ФИО автора
{%
    \texorpdfstring{% \texorpdfstring takes two arguments and uses the first for (La)TeX and the second for pdf
        \thesisAuthorLastName~\thesisAuthorOtherNames% так будет отображаться на титульном листе или в тексте, где будет использоваться переменная
    }{%
        \thesisAuthorLastName, \thesisAuthorOtherNames% эта запись для свойств pdf-файла. В таком виде, если pdf будет обработан программами для сбора библиографических сведений, будет правильно представлена фамилия.
    }
}
\newcommand{\thesisAuthorShort}        % Диссертация, ФИО автора инициалами
{\thesisAuthorInitials~\thesisAuthorLastName}
%\newcommand{\thesisUdk}                % Диссертация, УДК
%{\todo{xxx.xxx}}
\newcommand{\thesisTitle}              % Диссертация, название
{\todo{Оптические частотные гребенки и солитоны в микрорезонаторах}}
\newcommand{\thesisSpecialtyNumber}    % Диссертация, специальность, номер
{\todo{01.04.01}}
\newcommand{\thesisSpecialtyTitle}     % Диссертация, специальность, название
{\todo{Приборы и методы экспериментальной физики}}
\newcommand{\thesisDegree}             % Диссертация, ученая степень
{\todo{кандидата физико-математических наук}}
\newcommand{\thesisDegreeShort}        % Диссертация, ученая степень, краткая запись
{\todo{канд. физ.-мат. наук}}
\newcommand{\thesisCity}               % Диссертация, город написания диссертации
{\todo{Москва}}
\newcommand{\thesisYear}               % Диссертация, год написания диссертации
{\todo{2019}}
\newcommand{\thesisOrganization}       % Диссертация, организация
{\todo{МОСКОВСКИЙ ГОСУДАРСТВЕННЫЙ УНИВЕРСИТЕТ\\имени М.В. ЛОМОНОСОВА\\Физический факультет}}
\newcommand{\thesisOrganizationShort}  % Диссертация, краткое название организации для доклада
{\todo{НазУчДисРаб}}

\newcommand{\thesisInOrganization}     % Диссертация, организация в предложном падеже: Работа выполнена в ...
{\todo{МОСКОВСКОМ ГОСУДАРСТВЕННОМ УНИВЕРСИТЕТЕ имени М. В. ЛОМОНОСОВА}}

\newcommand{\supervisorFio}            % Научный руководитель, ФИО
{\todo{Биленко Игорь Антонович}}
\newcommand{\supervisorRegalia}        % Научный руководитель, регалии
{\todo{доктор физ.--мат. наук, профессор}}
\newcommand{\supervisorFioShort}       % Научный руководитель, ФИО
{\todo{И.\,А.~Биленко}}
\newcommand{\supervisorRegaliaShort}   % Научный руководитель, регалии
{\todo{д.ф.-м.н,~проф.}}


\newcommand{\opponentOneFio}           % Оппонент 1, ФИО
{\todo{Мурзина Татьяна Владимировна}}
\newcommand{\opponentOneRegalia}       % Оппонент 1, регалии
{\todo{доктор физико-математических наук}}
\newcommand{\opponentOneJobPlace}      % Оппонент 1, место работы
{\todo{Федеральное государственное бюджетное образовательное учреждение высшего образования Московский государственный университет имени М.В.Ломоносова. Физический факультет}}
\newcommand{\opponentOneJobPost}       % Оппонент 1, должность
{\todo{доцент}}

\newcommand{\opponentTwoFio}           % Оппонент 2, ФИО
{\todo{Карташов Ярослав Вячеславович}}
\newcommand{\opponentTwoRegalia}       % Оппонент 2, регалии
{\todo{доктор физико-математических наук}}
\newcommand{\opponentTwoJobPlace}      % Оппонент 2, место работы
{\todo{Федеральное государственное бюджетное учреждение науки Институт Спектроскопии РАН}}
\newcommand{\opponentTwoJobPost}       % Оппонент 2, должность
{\todo{ведущий научный сотрудник}}

\newcommand{\opponentTriFio}           % Оппонент 3, ФИО
{\todo{Величанский Владимир Леонидович}}
\newcommand{\opponentTriRegalia}       % Оппонент 3, регалии
{\todo{кандидат физико-математических наук}}
\newcommand{\opponentTriJobPlace}      % Оппонент 3, место работы
{\todo{Федеральное государственное бюджетное учреждение науки Физический институт им. П.Н. Лебедева РАН}}
\newcommand{\opponentTriJobPost}       % Оппонент 3, должность
{\todo{ведущий научный сотрудник}}


\newcommand{\leadingOrganizationTitle} % Ведущая организация, дополнительные строки
{\todo{Федеральное государственное бюджетное образовательное учреждение высшего профессионального образования с~длинным длинным длинным длинным названием}}

\newcommand{\defenseDate}              % Защита, дата
{\todo{30 мая 2019~г.~в~14-30~часов}}
\newcommand{\defenseCouncilNumber}     % Защита, номер диссертационного совета
{\todo{МГУ.01.12}}
\newcommand{\defenseCouncilTitle}      % Защита, учреждение диссертационного совета
{\todo{Москва}}
\newcommand{\defenseCouncilAddress}    % Защита, адрес учреждение диссертационного совета
{\todo{119991, Москва, Ленинские горы, д.1, стр.2}}
\newcommand{\defenseCouncilPhone}      % Телефон для справок
{\todo{+7~(495)~939-25-47}}

\newcommand{\defenseSecretaryFio}      % Секретарь диссертационного совета, ФИО
{\todo{Карташов Игорь Николаевич}}
\newcommand{\defenseSecretaryRegalia}  % Секретарь диссертационного совета, регалии
{\todo{к.ф.-м.н.}}            % Для сокращений есть ГОСТы, например: ГОСТ Р 7.0.12-2011 + http://base.garant.ru/179724/#block_30000

\newcommand{\synopsisLibrary}          % Автореферат, название библиотеки
{\todo{Физического факультета МГУ им. М.В.Ломоносова}}
\newcommand{\synopsisDate}             % Автореферат, дата рассылки
{\todo{26 апреля 2019 года}}

% To avoid conflict with beamer class use \providecommand
\providecommand{\keywords}%            % Ключевые слова для метаданных PDF диссертации и автореферата
{}
