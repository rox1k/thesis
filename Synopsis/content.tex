
\section*{Общая характеристика работы}

\newcommand{\actuality}{\underline{\textbf{\actualityTXT}}}
\newcommand{\progress}{\underline{\textbf{\progressTXT}}}
\newcommand{\aim}{\underline{{\textbf\aimTXT}}}
\newcommand{\tasks}{\underline{\textbf{\tasksTXT}}}
\newcommand{\novelty}{\underline{\textbf{\noveltyTXT}}}
\newcommand{\influence}{\underline{\textbf{\influenceTXT}}}
\newcommand{\methods}{\underline{\textbf{\methodsTXT}}}
\newcommand{\defpositions}{\underline{\textbf{\defpositionsTXT}}}
\newcommand{\reliability}{\underline{\textbf{\reliabilityTXT}}}
\newcommand{\probation}{\underline{\textbf{\probationTXT}}}
\newcommand{\contribution}{\underline{\textbf{\contributionTXT}}}
\newcommand{\publications}{\underline{\textbf{\publicationsTXT}}}


{\actuality} Открытие в 2007 году оптических частотных гребенок в оптических микрорезонаторах \cite{DelHaye2007} и последующая демонстрация оптических временных солитонов \cite{Herr2014} с низким уровнем фазового шума в 2014 году вызвали новую волну интереса и исследований в области микрорезонаторов и частотной метрологии \cite{Kippenberg2011,Kippenbergeaan8083}. Оптическая частотная гребенка в микрорезонаторе представляет собой набор эквидистантных спектральных линий, получающихся каскадно при накачке микрорезонатора из нелинейного материала с помощью лазера непрерывной мощности. Керровские частотные гребенки имеют расстояния между линиями 5-1000 ГГц и позволяют достичь уровня минитюаризации и энергоэффективности устройств, труднодостижимого для гребенок, полученных с помощью фемтосекундных лазеров с синхронизации мод. В последние годы наблюдается быстрый и существенный прогресс в этой области. За прошедшее с их открытия время был проведен теоретический анализ и обширное численное моделирование богатой нелинейной динамики процесса формирования оптических гребенок. Солитонный режим оптических гребенок был продемонстрирован в кристаллических микрорезонаторах из $MgF_2$ \cite{Herr2014}, в интегральных резонаторах из $Si_3N_4$ \cite{Brasch2016}, $Si$ \cite{Yu2016}, $SiO_2$ \cite{Yi2015}, $LiNbO_3$ \cite{2018arXiv181209610H}. Ширина оптической гребенки из интегрального микрорезонатора достигла октавы \cite{Pfeiffer:17}, солитоны были продемонстрированы в ближнем ИК \cite{Karpov2018}, телекоммуникационном диапазоне \cite{Herr2014} и среднем ИК \cite{Griffith2016}. Оптические частотные гребенки в микрорезонаторах обладают богатой нелинейной динамикой и привели к наблюдению различных эффектов: генерация светлых солитонов \cite{Herr2014}, темных импульсов \cite{Xue2015}, излучение дисперсионных волн \cite{Brasch2016}, Рамановское самовоздействие и сдвиг частоты \cite{Karpov2016}, Стоксовы солитоны \cite{Yang2016stokes}, бризерные режимы \cite{Lucas2017breather}, солитонные кристаллы \cite{Cole2017crystals}. Были проведены эксперименты, демонстрирующие их разнообразные практические применения: калибровка эшелле-спектрографов для поиска экзопланет \cite{Obrzud2019}, прямая спектроскопия поглощения веществ методом двойной гребенки \cite{Suh2016}, оптические часы \cite{Papp2014}, как источник каналов в телекоммуникации с рекордной пропускной способностью \cite{MarinPalomo2017}, как компактный радиофотонный источник СВЧ сигнала с низким уровнем фазового шума \cite{Liang2015}, как источник импульсов для быстрого измерения расстояний с высокой точностью (ЛИДАР) \cite{Trocha887}, как полностью интегральный синтезатор оптических частот \cite{Spencer2018}.

Количество публикаций с 2014 по 2018 годы с ключевыми словам солитоны в оптических резонаторах составило около 600, из них несколько десятков в самых высокорейтинговых научных журналах. В случае преодоления нескольких нерешенных проблем, оптические частотные гребенки из микрорезонаторов могут стать востребованным коммерческим продуктом.

Несмотря на обширное изучение оптических частотных гребенок и солитонов в микрорезонаторах, остаются нерешенными важные задачи: генерации оптической гребенки шириной в октаву с межмодовым расстоянием <100 ГГц, доступном для детектирования современной электроникой и полная ее стабилизация; полная интеграция генератора гребенки на чип, включая лазер накачки; повышение энергетической эффективности процесса нелинейного преобразования частот, повышение мощности оптической гребенки; детерминированная генерация оптических гребенок с малыми шумами при нормальной дисперсии групповой скорости, детерминированная генерация односолитонных режимов как в одном, так одновременно и в нескольких резонаторах.


% {\progress}
% Этот раздел должен быть отдельным структурным элементом по
% ГОСТ, но он, как правило, включается в описание актуальности
% темы. Нужен он отдельным структурынм элемементом или нет ---
% смотрите другие диссертации вашего совета, скорее всего не нужен.

{\aim} данной работы является поиск методов генерации оптических частотных гребенок в микрорезонаторах при различной дисперсии групповой скорости на различных длинах волн накачки, экспериментальное получение солитонов в кристаллических микрорезонаторах и изучение их возможных приложений.

Для~достижения поставленной цели необходимо было решить следующие {\tasks}:
\begin{enumerate}
  \item Разработать численный метод моделирования динамики оптических частотных гребенок в микрорезонаторах.
  \item Разработать методику изготовления кристаллических микрорезонаторов с высокой добротностью.
  \item Создать экспериментальную установку для изучения свойств кристаллических микрорезонаторов.
  \item Экспериментально получить солитонный режим генерации керровской гребенки, изучить его свойства и продемонстрировать практические применения.
\end{enumerate}


{\novelty}
\begin{enumerate}
  \item На основе численного моделирования были указаны ограничения на ширину оптической гребенки в зависимости от дисперсии групповой скорости резонатора и на влияние эффекта нормального расщепления мод на возможность генерации солитонов.
  \item На основе численного моделирования был предложен оригинальный метод генерации оптических частотных гребенок при нормальной дисперсии групповой скорости в резонаторе при отклонении дисперсионного закона.
  \item На основе численного моделирования был предложен оригинальный метод генерации оптических частотных гребенок при нормальной дисперсии групповой скорости в резонаторе при использовании двухчастотной или амлитудно-модулированной накачки.
  \item Впервые была продемонстрирована методика изготовления идентичных по форме кристаллических микрорезонаторов высокой добротности с различием по диаметру в 1 мкм.
  \item Впервые продемонстрирована возможность одновременнной генерации солитонов в идентичных микрорезонаторах, расположенных на одном кристаллическом цилиндре.
  \item Впервые продемонстрирована возможность одновременнной генерации солитонов в одном резонаторе на разных семействах пространственных мод, распространяющихся как в одном, так и в противоположных направлениях. Продемонстрирована применимость метода для спектроскопии поглощения веществ.
  \item Впервые продемонстрирована возможность долгосрочной стабилизации частоты повторения солитона с помощью затягивания на линию амплитудной модуляции лазера
\end{enumerate}

{\influence} разработанной методики изготовления кристаллических микрорезонаторов заключается в возможности повторяемо изготавливать высокодобротные микрорезонаторы с заданными характеристиками. Экспериментальная демонстрация генерации двойных оптических гребенок в одном резонаторе на разных семействах мод имеет прямое приложение в устройствах спектроскопии поглощения и быстрого измерения расстояний. Экспериментальная демонстрация стабилизированного солитонного режима имеет прямое практическое применение как источника высокостабильного СВЧ сигнала на частоте повторения солитона.

{\methods} В работе использовались как общенаучные методы: анализ, наблюдение, сравнение, эксперимент, так и специальные методы численного компьютерного моделирования.

{\defpositions}
\begin{enumerate}
  \item Численное моделирование, основанное на решении системы уравнений для нелинейно связанных мод, дает точную картину процесса генерации оптических частотоных гребенок и солитонов в резонаторах из материалов с кубичной нелинейностью. Результаты численного моделирования очень хорошо совпадают с экспериментальными данными по режимам генерации оптических частотных гребенок и их свойствам.
  \item Использование двухчастотной или амплитудно-модулированной накачки с разницей частот, кратной области свободной дисперсии резонатора, приводит к возможности генерации темных солитоноподобных структур в микрорезонаторах с нормальной дисперсией групповой скорости.
  \item Разработанный метод изготовления кристаллических микрорезонаторов с помощью алмазного точения единственной точкой и полировки алмазными суспензиями позволяет воспроизводимо изготавливать идентичные по форме микрорезонаторы сверхвысокой добротности с различием по диаметру не более 1 мкм (разницей ОСД до 2 МГц).
  \item В двух микрорезонаторах из $MgF_2$ оптимизированной формы, изготовленных на одном кристаллическом цилиндре, возможна воспроизводимая одновременная генерация оптических временных солитонов, обладающих узкими СВЧ сигналами биений так, что разность частот повторений солитонов в этих двух микрорезонаторах не превышает 2 МГц.
  %\item В высокодобротных микрорезонаторах из $MgF_2$ возможно воспроизводимо генерировать оптические временные солитоны при аномальной дисперсии групповой скорости, обладающие узким СВЧ сигналом на частоте повторения.
  \item Высокодобротные кристаллические микрорезонаторы c МШГ из $MgF_2$ поддерживают одновременную генерацию оптических временных солитонов на разных пространственных семействах мод, распространяющихся как в одном, так и в противоположных направлениях. Результирующий СВЧ сигнал мультигетеродинирования двух оптических солитонов может использоваться для проведения спектроскопии поглощения веществ для оптических линий, находящихся в области спектрального покрытия солитонов.
\end{enumerate}

{\reliability} полученных результатов определяется адекватностью использованных физических моделей и математических методов, выбранных для решения поставленных задач, корректностью использованных приближений, а также
соответствием результатов теоретических и численных расчетов и экспериментальных данных, и не вызывает сомнений. Результаты находятся в соответствии с результатами, полученными другими авторами.

Численная модель основана на решении системы нелинейных дифференциальных уравнений, полученных из уравнений Максвелла. Использовались широко известные численные методы решения ОДУ. В основе экспериментальных исследований лежали классические методы оптики и методики измерений физических величин.

{\probation}
Основные результаты работы докладывались~на:
\begin{enumerate}
  \item Microresonator Frequency Combs and Applications, Аскона, Швейцария 2014
  \item CLEO/QELS US, Сан-Хосе, США 2014
  \item PQE-2015, Сноуберд, США 2015
  \item CLEO US, Сан-Хосе, США 2015
  \item XV Всероссийская школа-семинар "Физика и применение микроволн" имени профессора А.П. Сухорукова ( "Волны-2015"), Красновидово, Россия 2015
  \item Third International Conference on Quantum Technologies (ICQT 2015), Москва, Россия 2015
  \item SPIE Photonics West, Сан-Франциско, США 2016
  \item 2016 International Conference Laser Optics (ICLO)  Санкт-Петербург, Россия 2016
  \item SPIE Photonics West, Сан-Франциско, США 2017
  \item Progress In Electromagnetics Research Symposium (PIERS 2017 in St Petersburg), Санкт-Петербург, Россия 2017
  \item XVI Всероссийская школа-семинар "Физика и применение микроволн" имени профессора А.П. Сухорукова ( "Волны-2017"), Красновидово, Россия 2017
  \item CLEO Europe and EQEC, Мюнхен, Германия 2017
  \item XVII Всероссийская школа-семинар "Физика и применение микроволн" имени профессора А.П. Сухорукова ( "Волны-2018"), Красновидово, Россия 2018
  \item Photonics West, Сан-Франциско, США 2018
  \item CLEO US, Сан-Хосе, США 2018
  \item 2018 International Conference Laser Optics (ICLO)  Санкт-Петербург, Россия 2018
  \item European Frequency and Time Forum, Турин, Италия 2018
  \item CLEO Pacific Rim 2018, Гонконг 2018
\end{enumerate}


{\contribution} Все результаты, вошедшие в диссертацию, получены либо лично
автором, либо совместно с соавторами работ, опубликованных по теме диссертации.

%\publications\ Основные результаты по теме диссертации изложены в ХХ печатных изданиях~\cite{Sokolov,Gaidaenko,Lermontov,Management},
%Х из которых изданы в журналах, рекомендованных ВАК~\cite{Sokolov,Gaidaenko},
%ХХ --- в тезисах докладов~\cite{Lermontov,Management}.

\ifnumequal{\value{bibliosel}}{0}{% Встроенная реализация с загрузкой файла через движок bibtex8
    \publications\ Основные результаты по теме диссертации изложены в 20 печатных изданиях,
    11 из которых изданы в журналах, рекомендованных ВАК,
    9 "--- в тезисах докладов:%
    
    \begin{enumerate}
  \item Herr T., Brasch V., Jost J.D., Mirgorodskiy I., Lihachev G., Gorodetsky M.L., Kippenberg T.J. Mode spectrum and temporal soliton formation in optical microresonators // Phys. Rev. Lett. — 2014. — Vol. 113. — P. 123901
  \item Lobanov V.E., Lihachev G., Kippenberg T.J., Gorodetsky M.L. Frequency combs and platicons in optical microresonators with normal GVD // Opt. Express. — 2015. — Vol. 23. — Pp. 7713–772
  \item V.E. Lobanov, G. Lihachev, M.L. Gorodetsky. Generation of platicons and frequency combs in optical microresonators with normal GVD by modulated pump // EPL. — 2015. — Vol. 112. — P. 54008.
  \item Lobanov V.E., Lihachev G.V., Pavlov N.G. et al. Harmonization of chaos into a soliton in Kerr frequency combs // Optics Express. — 2016. — Vol. 24, no. 24.— Pp. 27382–27394.
  \item Brasch V., Geiselmann M., Herr T., Lihachev G., Pfeiffer M.H.P, Gorodetsky M.L., Kippenberg T.J. Photonic chip based optical frequency comb using soliton induced Cherenkov radiation // Science. — 2016. — Vol. 351, no. 6271. — Pp. 357–360.
  \item Guo H., Karpov M., Lucas E., Kordts A., Pfeiffer M.H.P, Brasch V., Lihachev G., Lobanov V.E., Gorodetsky M.L., Kippenberg T.J. Universal dynamics and deterministic switching of dissipative Kerr solitons in optical microresonators // Nature Phys. — 2017. — Vol. 13, no. 1. — P. 94–102
  \item Pavlov N.G.,Lihachev G., Koptyaev S. et al. Soliton dual frequency combs in crystalline microresonators // Optics Lett. — 2017. — Vol. 42, no. 3. — Pp. 514–517.
  \item M. Anderson, N. G. Pavlov, J. D. Jost, G. Lihachev, J. Liu, T. Morais, M. Zervas, M. L. Gorodetsky, T. J. Kippenberg Highly efficient coupling of crystalline microresonators to integrated photonic waveguides // Opt. Lett. — 2018. — Vol. 43, no. 9. — Pp. 2106–2109
  \item E. Lucas, G. Lihachev, R. Bouchand et al. Spatial multiplexing of soliton microcombs // Nature Photonics. — 2018. — Vol. 12, no. 11. — Pp. 699–705.
  \item N. G. Pavlov, S. Koptyaev, G. V. Lihachev et al. Narrow-linewidth lasing and soliton Kerr microcombs with ordinary laser diodes // Nature Photonics. — 2018. — Vol. 12, no. 11. — Pp. 694–698
  \item W. Weng, E. Lucas, G. Lihachev et al.  Spectral Purification of Microwave Signals with Disciplined Dissipative Kerr Solitons // Phys. Rev. Lett. — 2019. — Vol. 122. — P. 013902
    \end{enumerate}
     

}{% Реализация пакетом biblatex через движок biber
%Сделана отдельная секция, чтобы не отображались в списке цитированных материалов
    \begin{refsection}[vak,papers,conf]% Подсчет и нумерация авторских работ. Засчитываются только те, которые были прописаны внутри \nocite{}.
        %Чтобы сменить порядок разделов в сгрупированном списке литературы необходимо перетасовать следующие три строчки, а также команды в разделе \newcommand*{\insertbiblioauthorgrouped} в файле biblio/biblatex.tex
        \printbibliography[heading=countauthorvak, env=countauthorvak, keyword=biblioauthorvak, section=1]%
        \printbibliography[heading=countauthorconf, env=countauthorconf, keyword=biblioauthorconf, section=1]%
        \printbibliography[heading=countauthornotvak, env=countauthornotvak, keyword=biblioauthornotvak, section=1]%
        \printbibliography[heading=countauthor, env=countauthor, keyword=biblioauthor, section=1]%
        \nocite{%Порядок перечисления в этом блоке определяет порядок вывода в списке публикаций автора
                HerrPRL2014,
                Brasch2016,
                Lobanov2015,
                Lobanov2015epl,
                Lobanov2016,
                Pavlov2017,
                Karpov2017,
                Anderson:18,
                Lucas2018,
                Pavlov2018,
                PhysRevLett.122.013902
            }%
        \publications\ Основные результаты по теме диссертации изложены в~\arabic{citeauthor}~печатных изданиях,
        \arabic{citeauthorvak} из которых изданы в журналах, рекомендованных ВАК,
        \arabic{
        authorconf} "--- в~тезисах докладов.
    \end{refsection}
    %\begin{refsection}[vak,papers,conf]%Блок, позволяющий отобрать из всех работ автора наиболее значимые, и только их вывести в автореферате, но считать в блоке выше общее число работ
%        \printbibliography[heading=countauthorvak, env=countauthorvak, keyword=biblioauthorvak, section=2]%
%        \printbibliography[heading=countauthornotvak, env=countauthornotvak, keyword=biblioauthornotvak, section=2]%
%        \printbibliography[heading=countauthorconf, env=countauthorconf, keyword=biblioauthorconf, section=2]%
%        \printbibliography[heading=countauthor, env=countauthor, keyword=biblioauthor, section=2]%
%        \nocite{vakbib2}%vak
%        \nocite{bib1}%notvak
%        \nocite{confbib1}%conf
%    \end{refsection}
}
%При использовании пакета \verb!biblatex! для автоматического подсчёта
%количества публикаций автора по теме диссертации, необходимо
%их~здесь перечислить с использованием команды \verb!\nocite!.
 % Характеристика работы по структуре во введении и в автореферате не отличается (ГОСТ Р 7.0.11, пункты 5.3.1 и 9.2.1), потому её загружаем из одного и того же внешнего файла, предварительно задав форму выделения некоторым параметрам

%Диссертационная работа была выполнена при поддержке грантов ...

%\underline{\textbf{Объем и структура работы.}} Диссертация состоит из~введения, четырех глав, заключения и~приложения. Полный объем диссертации \textbf{ХХХ}~страниц текста с~\textbf{ХХ}~рисунками и~5~таблицами. Список литературы содержит \textbf{ХХX}~наименование.

%\newpage
\section*{Содержание работы}
Во \underline{\textbf{введении}} обосновывается актуальность
исследований, проводимых в~рамках данной диссертационной работы,
приводится обзор научной литературы по изучаемой проблеме,
формулируется цель, ставятся задачи работы, излагается научная новизна
и практическая значимость представляемой работы.


\underline{\textbf{Первая глава}} посвящена обзору литературы, посвященной генерации оптических частотных гребенок и солитонов в микрорезонаторах. Над развитием этой новой области работают около 10 экспериментальных групп и еще несколько теоретических групп. За 11 прошедших с открытия лет был проведен обширный теоретический анализ и численное моделирование богатой нелинейной динамики процесса формирования оптических гребенок, были проведены эксперименты, демонстрирующие их фундаментальные свойства и важнейшие практически применения. Дан обзор многочисленных работ посвященных теоретическому и численному моделированию динамики генерации оптических гребенок, как шумных, так и высококогерентных. Рассмотрены работы посвященные экспериментальной демонстрации оптических гребенок и солитонов в кристаллических, интегральных резонаторах из различных материалов при накачке на длинах волн от ближнего ИК до среднего ИК. Сделан обзор экспериментальной демонстрации различных эффектов: излучения дисперсионной волны, Рамановского сдвига центра солитона, бризерных режимов, Стоксовы солитоны, различным методам контроля дисперсии резонатора.

Наблюдение диссипативных керровских солитонов в микрорезонаторах вызвало не только интерес к изучению богатой нелинейной физики солитонов, но и демонстрации множества применений в высокоточной метрологии и других технологиях. Даны обзоры продемонстрированных применений оптических частотных гребенок и солитонов: оптические часы, перенос точности СВЧ стандарта частоты в оптический диапазон, калибровка астрономических эшелле-спектрографов, прямая спектроскопия поглощения веществ с использованием двух оптических гребенок, в роли оптического источника в установке для спектрального уплотнения телекоммуникационных каналов и когерентной передачи данных, источником СВЧ сигнала с низким уровнем фазового шума, для быстрого измерения расстояний (ЛИДАР), интегральный синтезатор оптических частот.

\underline{\textbf{Вторая глава}} посвящена численному моделированию динамики генерации оптических гребенок и исследованию возможных режимов и поиску оптимальных параметров.

Представлен вывод системы уравнений для связанных мод из уравнений Максвелла в оптически нелинейном микрорезонаторе. Показана эквивалентность этой модели уравнению Луджиато-Лефевера. Это уравнение получается из нелинейного уравнения Шредингера (НУШ) добавлением слагаемых, отвечающих за диссипацию и накачку. НУШ является интегрируемой системой, для поиска решений которой используется метод обратной задачи рассеяния. Уравнение Луджиато-Лефевера не является интегрируемой системой, необходимо численное моделирование.

Далее описаны используемые численные методы с их сравнением: метод Рунге-Кутты с адаптивным шагом и оптимизацией для быстрого вычисления нелинейной суммы, а также Фурье метод расщепления по параметрам для численного решения уравнения Луджиато-Лефевера. Был реализован удобный программный интерфейс в среде Matlab для моделирования обоими методами при различной настройке всех параметров.

По результатам моделирования были выявлены параметры и режимы, при которых возможно достижение солитонного режима. Был учтен нагрев резонатора, различная связь и мощность накачки, а также ключевой параметр - отстройка частоты лазера накачки от моды микрорезонатора. Именно найденный при численном моделировании метод сканирования частоты лазера накачки, позволил экспериментально обнаружить солитоны в микрорезонаторах. Было проведено моделирование при различных дисперсионных законах, численно получена генерация дисперсионной волны (оптического аналога Черенковского излучения), мощность и положение спектрального пика при этом хорошо совпала с экспериментальными значениями. Выявлены диапазоны дисперсионных параметров, препятствующие образованию солитонов, а также влияние эффекта нормального расщепления мод вблизи моды накачки на возможность генерации солитонов. Результаты моделирования хорошо совпали с экспериментальными данными из литературы.

С помощью численного моделирование динамики оптических гребенок в микрорезонаторах с нормальной дисперсией групповой, было найдено два метода генерации солитоноподобных локализованных структур: локальное изменение дисперсионного закона вблизи моды накачки или использование двухчастотной или амплитудно-модулированной накачки с частотой модуляции строго равной области свободной дисперсии микрорезонатора. В обоих случаях были исследованы области существования и мягкого возбуждения оптических гребенок из вакуумных флуктуаций в модах резонатора, определены зависимости от величины дисперсии, мощности накачки, глубины модуляции или величины сдвига моды накачки по частоте. Показано значительное увеличение мощности оптической гребенки в режиме нормальной дисперсии. Предложенный метод был позже реализован экспериментально в других группах.

\underline{\textbf{Третья глава}} посвящена экспериментальному исследованию оптических частотных гребенок в кристаллических микрорезонаторах.

Разработана методика изготовления кристаллических микрорезонаторов методом алмазного точения с последующей полировкой алмазными суспензиями. Даны практические замечания по изготовлению кристаллических микрорезонаторов. В ходе работы были изготовлены резонаторы с добротностью не менее $10^8$ из кристаллических материалов $MgF_2,BaF_2,CaF_2,LiNbO_3$ и с меньшей добротностью из других материалов: $LiTaO_3,SiO_2,TGG,YLiF$. Методика позволяет воспроизводимо изготавливать микрорезонаторы с заданной геометрией с точность до 1 мкм. Минимальный диаметр изготовленных резонаторов составил 100 мкм, максимальный около 2 см. Были изготовлены резонаторы с микровыступами: оптическое поле было локализовано в выступах на образующей цилиндра размером 5 на 10 мкм.

Дано описание экспериментальной установки, использующейся в большинстве экспериментов. Экспериментально продемонстрирована генерация шумных оптических частотных в резонаторах из $MgF_2,BaF_2$ и солитонного режима в $MgF_2$/ Измерены термооптические мод в резонаторе из $BaF_2$, в этом же материале наблюдались эффекты вынужденного комбинационного рассеяния и вынужденного рассеяния Мандельштама-Бриллюэна при накачке на длине волны 1550 нм, определены из частоты и ширины соответствующих сигналов биений.

Продемонстрирован солитонный режим генерации гребенки, показан метод достижения односолитонного режима с помощью амплитудной модуляции накачки на частоте повторения солитона. Также этим же методом показана стабилизация частоты повторения солитона из-за эффекта затягивания со значительным улучшением стабильности на больших временах.

%Можно сослаться на свои работы в автореферате. Для этого в файле
%\verb!Synopsis/setup.tex! необходимо присвоить положительное значение
%счётчику \verb!\setcounter{usefootcite}{1}!. В таком случае ссылки на
%работы других авторов будут подстрочными.
%\ifnumgreater{\value{usefootcite}}{0}{
%Изложенные в третьей главе результаты опубликованы в~\cite{vakbib1, vakbib2}.
%}{}
%Использование подстрочных ссылок внутри таблиц может вызывать проблемы.

В \underline{\textbf{четвертой главе}} приведены методы генерации двойных оптических гребенок и солитонов в кристаллических микрорезонаторах.

Экспериментально показана генерация двух солитонных оптических гребенок в двух резонаторах на одном цилиндре.

Экспериментально продемонстрирована генерация солитонных оптических гребенок в одном резонаторе на разных семействах мод в одном направлении.

Экспериментально продемонстрирована генерация солитонных оптических гребенок в одном резонаторе на разных семействах мод в противоположных направлениях. Продемонстрировано применение метода для прямой спектроскопии поглощения веществ.

Экспериментально продемонстрирована генерация солитонных оптических гребенок в одном резонаторе на одном семействе мод в противоположных направлениях

В \underline{\textbf{заключении}} приведены основные результаты работы, которые заключаются в следующем:
%% Согласно ГОСТ Р 7.0.11-2011:
%% 5.3.3 В заключении диссертации излагают итоги выполненного исследования, рекомендации, перспективы дальнейшей разработки темы.
%% 9.2.3 В заключении автореферата диссертации излагают итоги данного исследования, рекомендации и перспективы дальнейшей разработки темы.
\begin{enumerate}
  \item Численные исследования модели уравнений связанных мод показали, что в высокодобротных микрорезонаторах возможна генерация оптических солитонов, были изучены диапазоны параметров и условия, влияющие на их эффективную генерацию.
  \item Математическое моделирование показало, что при нормальной дисперсии групповой скорости резонатора возможна генерация темных солитоноподобных структур при условии отличия закона дисперсии от параболического вида рядом с модой накачки или при использовании двухчастотной или амплитудно-модулированной накачки.
  \item Для выполнения экспериментальных исследований была разработана методика изготовления кристаллических микрорезонаторов методом алмазного точения и полировки алмазными суспензиями. Она позволила изготовить высокодобротные резонаторы из различных материалов с оптимизированной геометртей для работы с различными элементами связи.
  \item Экспериментально была продемонстрирована генерация оптических солитонов в резонаторах из MgF$_2$ с частотами повторений от $8.5$ до $27$ ГГц. При активной стабилизации температуры и отстройки частоты лазера накачки солитон существовал длительное время (до 3 часов), достаточное для экспериментальной демонстрации применений.
  \item Впервые продемонстрирована возможность одновременной генерации солитонов в идентичных микрорезонаторах, расположенных на одном кристаллическом цилиндре.
  \item Впервые продемонстрирована возможность одновременной генерации солитонов в одном резонаторе на разных семействах пространственных мод, распространяющихся как в одном, так и в противоположных направлениях. Показана применимость метода для спектроскопии поглощения веществ.
  \item Продемонстрирован метод стабилизации частоты повторения солитона с помощью эффекта захватывания на частоту амплитудной модуляции лазера накачки.
  %\item Для дальнейших исследований в данной области важнейшей задачей является демонстрация гребенок, шириной в октаву, в солитонном режиме с межмодовым расстоянием, доступном для детектирования современной электроникой, а также оптических частотных гребенок в резонаторах с нормальной дисперсией. Из новых практических применений возможна демонстрация фотонного АЦП с использованием двойных оптических гребенок.
\end{enumerate}



%\newpage
При использовании пакета \verb!biblatex! список публикаций автора по теме
диссертации формируется в разделе <<\publications>>\ файла
\verb!../common/characteristic.tex!  при помощи команды \verb!\nocite!

\ifdefmacro{\microtypesetup}{\microtypesetup{protrusion=false}}{} % не рекомендуется применять пакет микротипографики к автоматически генерируемому списку литературы
\ifnumequal{\value{bibliosel}}{0}{% Встроенная реализация с загрузкой файла через движок bibtex8
  \renewcommand{\bibname}{\large \authorbibtitle}
  \nocite{*}
  \insertbiblioauthor           % Подключаем Bib-базы
  %\insertbiblioother   % !!! bibtex не умеет работать с несколькими библиографиями !!!
}{% Реализация пакетом biblatex через движок biber
  \ifnumgreater{\value{usefootcite}}{0}{
%  \nocite{*} % Невидимая цитата всех работ, позволит вывести все работы автора
  \insertbiblioauthorcited      % Вывод процитированных в автореферате работ автора
  }{
  \insertbiblioauthor           % Вывод всех работ автора
%  \insertbiblioauthorgrouped    % Вывод всех работ автора, сгруппированных по источникам
%  \insertbiblioauthorimportant  % Вывод наиболее значимых работ автора (определяется в файле characteristic во второй section)
  \insertbiblioother            % Вывод списка литературы, на которую ссылались в тексте автореферата
  }
}
\ifdefmacro{\microtypesetup}{\microtypesetup{protrusion=true}}{}

